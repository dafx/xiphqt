\section{Decoding audio data}
\label{group__decode}\index{Decoding audio data@{Decoding audio data}}
To decode audio data using libfishsound:.  
\begin{itemize}
\item create a Fish\-Sound$\ast$ object with mode FISH\_\-SOUND\_\-DECODE. {\bf fish\_\-sound\_\-new()}{\rm (p.\,\pageref{fishsound_8h_a4})} will return a new Fish\-Sound$\ast$ object, initialised for decoding, and the {\bf Fish\-Sound\-Info}{\rm (p.\,\pageref{structFishSoundInfo})} structure will be cleared.\item provide a Fish\-Sound\-Decoded callback for libfishsound to call when it has decoded audio.\item (optionally) specify whether you want to receive interleaved or per-channel PCM data, using a {\bf fish\_\-sound\_\-set\_\-interleave()}{\rm (p.\,\pageref{fishsound_8h_a14})}. The default is for per-channel (non-interleaved) PCM.\item feed encoded audio data to libfishsound via {\bf fish\_\-sound\_\-decode()}{\rm (p.\,\pageref{fishsound_8h_a7})}. libfishsound will decode the audio for you, calling the Fish\-Sound\-Decoded callback you provided earlier each time it has a block of audio ready.\item when finished, call {\bf fish\_\-sound\_\-delete()}{\rm (p.\,\pageref{fishsound_8h_a11})}.\end{itemize}


This procedure is illustrated in src/examples/fishsound-decode.c. Note that this example additionally:\begin{itemize}
\item uses {\tt liboggz} to demultiplex audio data from an Ogg encapsulated Vorbis or Speex stream. Hence, the step of feeding encoded data to libfishsound is done within the Oggz\-Read\-Packet callback.\item uses {\tt libsndfile} to write the decoded audio to a WAV file.\end{itemize}


Hence this example code demonstrates all that is needed to decode both Ogg Vorbis and Ogg Speex files:



\footnotesize\begin{verbatim}/*
   Copyright (C) 2003 Commonwealth Scientific and Industrial Research
   Organisation (CSIRO) Australia

   Redistribution and use in source and binary forms, with or without
   modification, are permitted provided that the following conditions
   are met:

   - Redistributions of source code must retain the above copyright
   notice, this list of conditions and the following disclaimer.

   - Redistributions in binary form must reproduce the above copyright
   notice, this list of conditions and the following disclaimer in the
   documentation and/or other materials provided with the distribution.

   - Neither the name of CSIRO Australia nor the names of its
   contributors may be used to endorse or promote products derived from
   this software without specific prior written permission.

   THIS SOFTWARE IS PROVIDED BY THE COPYRIGHT HOLDERS AND CONTRIBUTORS
   ``AS IS'' AND ANY EXPRESS OR IMPLIED WARRANTIES, INCLUDING, BUT NOT
   LIMITED TO, THE IMPLIED WARRANTIES OF MERCHANTABILITY AND FITNESS FOR A
   PARTICULAR PURPOSE ARE DISCLAIMED.  IN NO EVENT SHALL THE ORGANISATION OR
   CONTRIBUTORS BE LIABLE FOR ANY DIRECT, INDIRECT, INCIDENTAL, SPECIAL,
   EXEMPLARY, OR CONSEQUENTIAL DAMAGES (INCLUDING, BUT NOT LIMITED TO,
   PROCUREMENT OF SUBSTITUTE GOODS OR SERVICES; LOSS OF USE, DATA, OR
   PROFITS; OR BUSINESS INTERRUPTION) HOWEVER CAUSED AND ON ANY THEORY OF
   LIABILITY, WHETHER IN CONTRACT, STRICT LIABILITY, OR TORT (INCLUDING
   NEGLIGENCE OR OTHERWISE) ARISING IN ANY WAY OUT OF THE USE OF THIS
   SOFTWARE, EVEN IF ADVISED OF THE POSSIBILITY OF SUCH DAMAGE.
*/

#include "config.h"

#include <stdio.h>
#include <stdlib.h>
#include <string.h>

#include <oggz/oggz.h>
#include <fishsound/fishsound.h>
#include <sndfile.h>

static char * infilename, * outfilename;
static int begun = 0;
static FishSoundInfo fsinfo;
static SNDFILE * sndfile;

static int
open_output (int samplerate, int channels)
{
  SF_INFO sfinfo;

  sfinfo.samplerate = samplerate;
  sfinfo.channels = channels;
  sfinfo.format = SF_FORMAT_WAV | SF_FORMAT_PCM_16;

  sndfile = sf_open (outfilename, SFM_WRITE, &sfinfo);

  return 0;
}

static int
decoded (FishSound * fsound, float ** pcm, long frames, void * user_data)
{
  if (!begun) {
    fish_sound_command (fsound, FISH_SOUND_GET_INFO, &fsinfo,
                        sizeof (FishSoundInfo));
    open_output (fsinfo.samplerate, fsinfo.channels);
    begun = 1;
  }

  sf_writef_float (sndfile, (float *)pcm, frames);

  return 0;
}

static int
read_packet (OGGZ * oggz, ogg_packet * op, long serialno, void * user_data)
{
  FishSound * fsound = (FishSound *)user_data;

  fish_sound_prepare_truncation (fsound, op->granulepos, op->e_o_s);
  fish_sound_decode (fsound, op->packet, op->bytes);

  return 0;
}

int
main (int argc, char ** argv)
{
  OGGZ * oggz;
  FishSound * fsound;
  long n;

  if (argc < 3) {
    printf ("usage: %s infilename outfilename\n", argv[0]);
    printf ("*** FishSound example program. ***\n");
    printf ("Decodes an Ogg Speex or Ogg Vorbis file producing a PCM wav file.\n");
    exit (1);
  }

  infilename = argv[1];
  outfilename = argv[2];

  fsound = fish_sound_new (FISH_SOUND_DECODE, &fsinfo);

  fish_sound_set_interleave (fsound, 1);

  fish_sound_set_decoded_callback (fsound, decoded, NULL);

  if ((oggz = oggz_open ((char *) infilename, OGGZ_READ)) == NULL) {
    printf ("unable to open file %s\n", infilename);
    exit (1);
  }

  oggz_set_read_callback (oggz, -1, read_packet, fsound);

  while ((n = oggz_read (oggz, 1024)) > 0);

  oggz_close (oggz);

  fish_sound_delete (fsound);
  
  sf_close (sndfile);

  exit (0);
}

\end{verbatim}
\normalsize
 