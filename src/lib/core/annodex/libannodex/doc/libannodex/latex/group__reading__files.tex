\section{Reading from files and file descriptors}
\label{group__reading__files}\index{Reading from files and file descriptors@{Reading from files and file descriptors}}
If the Annodex media you wish to access is directly available as a local file or via a file descriptor (such as a network socket), it can be directly opened as follows:

\begin{itemize}
\item open an annodex using {\bf anx\_\-open()}{\rm (p.\,\pageref{anx__general_8h_a3})} or anx\_\-openfd()\item attach read callbacks using anx\_\-set\_\-read\_\-$\ast$\_\-callback()\item call {\bf anx\_\-read()}{\rm (p.\,\pageref{anx__read_8h_a16})} repeatedly until it returns 0 or -1\item close the annodex with {\bf anx\_\-close()}{\rm (p.\,\pageref{anx__general_8h_a7})}\end{itemize}


This procedure is illustrated in src/examples/print-title-file.c:



\footnotesize\begin{verbatim}
#include <stdio.h>
#include <annodex/annodex.h>

static int
read_head (ANNODEX * anx, const AnxHead * head, void * user_data)
{
  puts (head->title);
  return ANX_CONTINUE;
}

int
main (int argc, char *argv[])
{
  ANNODEX * anx = NULL;
  char * filename;
  long n;

  if (argc != 2) {
    fprintf (stderr, "Usage: %s file.anx\n", argv[0]);
    exit (1);
  }

  filename = argv[1];

  anx = anx_open (filename, ANX_READ);

  anx_set_read_head_callback (anx, read_head, NULL);

  while ((n = anx_read (anx, 1024)) > 0);

  anx_close (anx);

  exit (0);
}
\end{verbatim}
\normalsize
 

