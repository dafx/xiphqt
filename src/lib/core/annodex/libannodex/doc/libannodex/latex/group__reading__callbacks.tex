\section{Advanced management of Anx\-Read callbacks}
\label{group__reading__callbacks}\index{Advanced management of AnxRead callbacks@{Advanced management of AnxRead callbacks}}
You retain control of the number of bytes read or input, and the callbacks you provide can instruct libannodex to immediately return control back to your application.\subsection{Callbacks}\label{rcb_list}
It is not required to implement all callbacks.\subsection{Return values}\label{rcb_retvals}
\begin{itemize}
\item ANX\_\-CONTINUE on success, and to inform anx\_\-read$\ast$() functions to continue on to the next packet\item ANX\_\-STOP\_\-OK on success, to inform anx\_\-read$\ast$() functions to return without further processing\item ANX\_\-STOP\_\-ERR on error, to inform anx\_\-read$\ast$() functions to return without further processing\end{itemize}


These mechanisms are illustrated in src/examples/print-lots.c:



\footnotesize\begin{verbatim}
#include <stdio.h>
#include <string.h>
#include <annodex/annodex.h>

struct my_data {
  char * filename;
  long interesting_serialno;
  int interesting_raw_packets;
  int done;
};

static int
read_stream (ANNODEX * anx, double timebase, char * utc, void * user_data)
{
  struct my_data * happy = (struct my_data *) user_data;

  printf ("Welcome to %s! The timebase is %f\n", happy->filename, timebase);
  return ANX_CONTINUE;
}

static int
read_track (ANNODEX * anx, long serialno, char * id, char * content_type,
            anx_int64_t granule_rate_n, anx_int64_t granule_rate_d,
            int nr_header_packets, void * user_data)
{
  struct my_data * happy = (struct my_data *) user_data;

  /* Ignore the annotations track, we don't find it interesting! */
  if (!strncmp (content_type, "text/x-cmml", 12)) return ANX_CONTINUE;

  printf ("Our first track has content-type %s and granule rate %ld/%ld.\n",
          content_type, (long)granule_rate_n, (long)granule_rate_d);

  printf ("We will remember it by its serial number %ld "
          "and mark it with crosses.\n", serialno);
  happy->interesting_serialno = serialno;

  /* We don't care about any other tracks! */
  anx_set_read_track_callback (anx, NULL, NULL);

  return ANX_CONTINUE;
}

static int
read_raw (ANNODEX * annodex, unsigned char * buf, long n,
            long serialno, anx_int64_t granulepos, void * user_data)
{
  struct my_data * happy = (struct my_data *) user_data;

  if (happy->done) {
    putchar ('!');
  } else if (serialno == happy->interesting_serialno) {
    happy->interesting_raw_packets++;
    putchar ('+');
  } else {
    putchar ('.');
  }

  return ANX_CONTINUE;
}

static int
read_clip3 (ANNODEX * anx, const AnxClip * clip, void * user_data)
{
  struct my_data * happy = (struct my_data *) user_data;

  printf ("\nAnd the third clip links to %s\n", clip->anchor_href);

  happy->done = 1;

  printf ("This completes our whirlwind tour of the first three clips!\n");
  return ANX_STOP_OK;
}

static int
read_clip2 (ANNODEX * anx, const AnxClip * clip, void * user_data)
{
  printf ("\nThe second clip links to %s\n", clip->anchor_href);
  anx_set_read_clip_callback (anx, read_clip3, user_data);
  return ANX_CONTINUE;
}

static int
read_clip1 (ANNODEX * anx, const AnxClip * clip, void * user_data)
{
  printf ("\nThe first clip links to %s\n", clip->anchor_href);
  anx_set_read_clip_callback (anx, read_clip2, user_data);
  return ANX_CONTINUE;
}

int
main (int argc, char *argv[])
{
  ANNODEX * anx = NULL;
  struct my_data me;
  long n;

  if (argc != 2) {
    fprintf (stderr, "Usage: %s file.anx\n", argv[0]);
    exit (1);
  }

  me.filename = argv[1];
  me.interesting_serialno = -1;
  me.interesting_raw_packets = 0;
  me.done = 0;

  anx = anx_open (me.filename, ANX_READ);

  anx_set_read_stream_callback (anx, read_stream, &me);
  anx_set_read_track_callback (anx, read_track, &me);
  anx_set_read_raw_callback (anx, read_raw, &me);
  anx_set_read_clip_callback (anx, read_clip1, &me);

  while (!me.done && (n = anx_read (anx, 1024)) > 0);

  printf ("%d packets from the first track (serialno %ld) were received\n",
          me.interesting_raw_packets, me.interesting_serialno);
  printf ("before the third clip.\n");

  anx_close (anx);

  exit (0);
}
\end{verbatim}
\normalsize


which produces output like: \small\begin{alltt}
Welcome to /tmp/alien\_song.anx! The timebase is 20.000000
Our first track has mime-type video/x-theora and granule rate 1536/1.
We will remember it by its serial number 1810353996 and mark it with crosses.
+...+++++++++++++++++...................................++++++++++++++..........
......................++++++++++++..............................++++++++++++++++
++++..............................+++++++++++++.................................
...++++++++++++++++++++++................................++++++++++++++.........
........................++++++++++++++++....................................++++
++++++++++.......................................++++++++++++++.................
......+++++++++++++.......................+++++++++++++.........................
...+++++++++++++..................................+++++++++++++.................
......+++++++++++++...............................++++++++++++++................
..........+++++++++++..........................++++++++++++++...................
.....++++++++++++.......................++++++++++++........................++++
++++++++........................+++++++++++++........................+++++++++++
+++++.......................+++++++++++.......................+++++++++++++++++.
......................+++++++++++++........................++++++++++...........
.............+++++++++++.......................++++++++++++++++++...............
........++++++++........................+++++++++++++++++.......................
++++++++.......................+++++++++++++++++........................++++++++
++++++...........................+++++++++++........................++++++++++++
+++........................+++++++++++.......................++++++++++++++.....
.................++++++++++++++..............................++++++++++.........
.........................++++++++++++................................+++++++++++
+++...................................++++++
The first anchor links to {\tt http://www.mars.int/}
+
The second anchor links to {\tt http://www.pluto.int/}
+++......................+++++++++++.............................+++++++++++....
...........................+++++++++++..........................................
.......+++++++++++++...........................+++++++++++......................
.......+\end{alltt}\normalsize 


\small\begin{alltt}And the third anchor links to {\tt http://www.venus.int/}
This completes our whirlwind tour of the first three anchors!\end{alltt}\normalsize 


\small\begin{alltt}639 packets from the first track (serialno 1810353996) were received
before the third anchor.\end{alltt}\normalsize 


\small\begin{alltt} \end{alltt}\normalsize 
 

