\section{anx\_\-int64.h File Reference}
\label{anx__int64_8h}\index{anx_int64.h@{anx\_\-int64.h}}


\subsection{Detailed Description}
Platform specific types for anx\_\-int64\_\-t. 

Adapted from libogg's method.

This file is included by {\bf $<$annodex/anx\_\-types.h$>$ }{\rm (p.\,\pageref{anx__types_8h})}, except on non-GNU Win32 systems {\bf }{\rm (p.\,\pageref{anx__int64__w32_8h})} is included instead.

This file should never be included directly by user code.

{\tt \#include $<$sys/types.h$>$}\par
\subsection*{Typedefs}
\begin{CompactItemize}
\item 
typedef int64\_\-t {\bf anx\_\-int64\_\-t}
\begin{CompactList}\small\item\em This typedef was determined on the system on which the documentation was generated. \item\end{CompactList}\end{CompactItemize}


\subsection{Typedef Documentation}
\index{anx_int64.h@{anx\_\-int64.h}!anx_int64_t@{anx\_\-int64\_\-t}}
\index{anx_int64_t@{anx\_\-int64\_\-t}!anx_int64.h@{anx\_\-int64.h}}
\subsubsection{\setlength{\rightskip}{0pt plus 5cm}typedef int64\_\-t {\bf anx\_\-int64\_\-t}}\label{anx__int64_8h_a0}


This typedef was determined on the system on which the documentation was generated. 

To query this on your system, do eg.

\small\begin{alltt}
   echo "#include <annodex/anx\_int64.h>" | gcc -E - | grep anx\_int64\_t
 \end{alltt}\normalsize 
 