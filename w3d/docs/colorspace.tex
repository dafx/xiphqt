
\section{ Colorspace Conversion }

\subsection{ exact YUV }

An Integer implementation of the exact RGB to YUV transform is given by 

\begin{eqnarray}
Y &=& ( 77 R + 150 G +  29 B) / 256    \nonumber\\
U &=& (-44 R -  87 G + 131 B) / 256    \nonumber\\
V &=& (131 R - 110 G -  21 B) / 256    \nonumber\\
\nonumber
\end{eqnarray}

You get the reversed transform by inverting this matrix. I don't have the 
integer representation here, simply multiply all values by 256/256 if you
really need one \dots

\begin{eqnarray}
R &=& Y + 1.371 V                      \nonumber\\
G &=& Y - 0.698 V - 0.336 U            \nonumber\\
B &=& Y + 1.732 U                      \nonumber\\
\nonumber
\end{eqnarray}

\subsection{ something like YUV but faster }

Since the wavelet transform is expansive, all computations have to be done using
at least 16bit Integers when transforming 8bit Integers. Thus we have to copy
the entire image into a buffer of 16bit Integers. The following transform is 
cheap enough to be done on-the-fly while copying data:

\begin{eqnarray}
V &=& R - G                                          \nonumber\\
U &=& B - G                                          \nonumber\\
Y &=& G + (U + V) / 4                                \nonumber\\
\nonumber
\end{eqnarray}

You can understand this transform as reversible (1,2) - wavelet transform on a row
of three data items. It is expanding (8 input bits require 9 output bits) because
of the subtraction.

As usual we can generate the inverse transform by running the algorithm
backwards and inverting all operations:

\begin{eqnarray}
G &=& Y - (U + V) / 4                                \nonumber\\
B &=& U + G                                          \nonumber\\
R &=& V + G                                          \nonumber\\
\nonumber
\end{eqnarray}

If someone needs as 'exact YUV to this colorspace' transform he can simply 
multiply the two transform matrices together.

