\section{Multitrack annotations in CMML}
\label{group__multicliptrack}\index{Multitrack annotations in CMML@{Multitrack annotations in CMML}}
The default case for CMML is to have one annotation track with temporally non-overlapping clips of annotations. This is the case when the \char`\"{}track\char`\"{} attribute of the {\bf clip} tag is not used.

It is however possible to specify several tracks of annotations within one CMML file. To that end, you need to give each track that you're using a name. This name needs to be written into the \char`\"{}track\char`\"{} attribute of the {\bf clip} tags which belong onto that track. Not providing a track attribute attaches the clip to the default track.

Take care that the clips in one annotation track do not overlap in time.

This is an example of such a multi-track annotation specification: 

\footnotesize\begin{verbatim}
<clip track="german" lang="de" id="intro_ge" start="npt:0">
  <a href="http://www.blah.au/fish.html">Lesen Sie mehr &#252;ber Fische.</a>
  <desc>Dies ist die Einleitung zum Film &#252;ber Fische von Joe.</desc>
</clip>

<clip track="default" lang="en" id="intro" start="npt:0">
  <a href="http://www.blah.au/fish.html">Read more about fish.</a>
  <desc>This is the introduction to the film Joe made about fish.</desc>
</clip>

<clip track="german" id="dolphin_ge" start="npt:3.5">
  <desc>Joe hat einen Delphin im Meer entdeckt.</desc>
  <meta name="Thema" content="Delphin"/>
</clip>

<clip id="dolphin" start="npt:3.5">
  <desc>Here, Joe caught sight of a dolphin in the ocean.</desc>
  <meta name="Subject" content="dolphin"/>
</clip>
\end{verbatim}
\normalsize
