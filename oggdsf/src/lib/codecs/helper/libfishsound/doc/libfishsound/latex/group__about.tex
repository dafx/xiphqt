\section{About}
\label{group__about}\index{About@{About}}
\subsection{Design}\label{group__about_design}
libfishsound provides a simple programming interface for decoding and encoding audio data using codecs from {\tt Xiph.Org}.

libfishsound by itself is designed to handle raw codec streams from a lower level layer such as UDP datagrams. When these codecs are used in files, they are commonly encapsulated in {\tt Ogg} to produce {\em Ogg Vorbis\/} and {\em Speex\/} files. Example C programs using {\tt liboggz} to read and write these files are provided in the libfishsound sources.

libfishsound is implemented as a wrapper around the existing codec libraries and provides a consistent, higher-level programming interface. The motivation for this is twofold: to simplify the task of developing application software that supports these codecs, and to ensure that valid codec streams are generated.\subsection{History}\label{group__about_history}
libfishsound was designed and developed by Conrad Parker on the weekend of October 18-19 2003. Previously the author had implemented Vorbis and Speex support in the following software:\begin{itemize}
\item {\tt Sweep}, a digital audio editor with decoding and GUI control of all encoding options of Vorbis and Speex\item Speex support in the {\tt xine} multimedia player\item Vorbis and Speex importers for {\tt libannodex}, the basic library for reading and writing {\tt Annodex.net} media files.\end{itemize}


The implementation of libfishsound draws heavily on these sources, and in turn the original example sources of libvorbis and libvorbisenc by Monty, and libspeex by Jean-Marc Valin.

The naming of libfishsound reflects both the Xiph.Org logo and the author's reputation as a dirty, smelly old fish.\subsection{Limitations}\label{group__about_limitations}
libfishsound has been designed to accomodate the various decoding and encoding styles required by a wide variety of software. However, as it is an abstraction of the underlying libvorbis, libvorbisenc and libspeex libraries, it may not be possible to implement some low-level techniques that these libraries enable, such as parallelization of Vorbis sub-block decoding. Nevertheless it is expected that libfishsound is a useful API for most software requiring Vorbis or Speex support, including most applications the author has encountered.\subsection{Acknowledgements}\label{group__about_acknowledgements}
Much of the API design follows the style of {\tt libsndfile}. The author would like to thank Erik de Castro Lopo for feedback on the design of libfishsound. 