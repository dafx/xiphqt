\documentclass[11pt,letterpaper]{article}

\usepackage{latexsym}
\usepackage{amssymb}
\usepackage{amsmath}
\usepackage{graphicx}
\usepackage{booktabs}
\usepackage[pdfpagemode=None,pdfstartview=FitH,pdfview=FitH,colorlinks=true]%
 {hyperref}

\newtheorem{theorem}{Theorem}[section]
\newcommand{\qi}{\ensuremath{\mathit{qi}} }
\newcommand{\ti}{\ensuremath{\mathit{ti}} }
\newcommand{\bitvar}[1]{\ensuremath{\left[\mathrm{#1}\right]}}
\newcommand{\term}[1]{{\em #1}}

\pagestyle{headings}
\bibliographystyle{alpha}

\title{Theora I Specification}
\author{Xiph.org Foundation}
\date{\today}

\begin{document}

\maketitle
\tableofcontents
\newpage

\section{Introduction and Description}

This section provides a high level description of the Theora codec's
 construction.
A bit-by-bit specification appears beginning in Section~\ref{sec:bitpacking}.
The later sections assume a high-level understanding of the Theora decode
 process, which is provided below.

\subsection{Overview}

Theora is a general purpose, lossy video codec.
It is based on the VP3 video codec produced by On2 Technologies
 (\url{http://www.on2.com/}).
On2 Technologies donated the VP3.2 source code to the Xiph.org
 Foundation and it was released under a BSD-like license.
On2 also made an irrevocable, royalty-free license grant for any patent claims
 it might have over the software and any derivatives.
No formal specification exists for the VP3 format beyond this source code,
 though Mike Melanson maintains a detailed description \cite{Mel04}.
Portions of this specification were adopted from his text with permission.

\subsubsection{VP3 and Theora}

Theora contains a superset of the features that were available in the original
 VP3 codec.
Content encoded with VP3.2 can be losslessly transcoded into the Theora format.
%TODO: what about VP3.1 etc? source tables all say 'VP31'
Theora content cannot, in general, be losslessly transcoded into the VP3
 format.
If a feature is not available in the original VP3 format, this is mentioned
 when that feature is defined.
A complete list of these features appears in Appendix~REF.

\subsubsection{Video Formats}

Theora I currently supports progressive video data of arbitrary dimensions at a
 constant frame rate in one of several $Y'C_bC_r$ color spaces.
The precise definition the color spaces supported appears in
 Section~\ref{sec:colorspaces}.
Three different chroma subsampling formats are supported: 4:2:0, 4:2:2,
 and 4:4:4.
The precise details of each of these formats and their sampling locations are
 described in Section~REF.

The Theora I format does not support interlaced material, variable frame rates,
 bit-depths larger than 8 bits per component, nor alternate color spaces such
 as RGB or arbitrary multi-channel spaces.
Black and white content can be efficiently encoded, however, because the
 uniform chroma planes compress well.
Support for interlaced material is planned for a future version.
Support for infrequently changing frame rates can already be achieved by
 chaining several Theora streams together.
Support for increased bit depths or additional color spaces is not planned.

\subsubsection{Classification}

Theora I is a block-based lossy transform codec that utilizes an
 $8\times 8$ Type-II Discrete Cosine Transform and block-based motion
 compensation.
This places it in the same class of codecs as MPEG-1, -2, -4, and H.263.
The details of how individual blocks are organized and how DCT coefficients are
 organized in the bitstream differ substantially from these codecs, however.
Theora supports only intra frames (I frames in MPEG) and inter frames (P frames
 in MPEG).
There is no equivalent to the bi-predictive frames (B frames) found in MPEG
 codecs.

\subsubsection{Assumptions}

The Theora codec design assumes a complex, psychovisually-aware encoder and a
 simple, low-complexity decoder.
%TODO: Talk more about implementation complexity.

Theora provides none of its own framing, synchronization, or protection against
 transmission errors; it is solely a method of accepting input video frames and
 compressing these frames into raw, unformatted `packets'.
The decoder then accepts these raw packets in sequence, decodes them, and
 synthesizes a fascimile of the original video frames.
Theora is a free-form variable bit rate (VBR) codec, and packets have no
 minimum size, maximum size, or fixed/expected size.

Theora packets are thus intended to be used with a transport mechanism that
 provides free-form framing, synchronization, positioning, and error correction
 in accordance with these design assumptions, such as Ogg (for file transport)
 or RTP (for network multicast).
For the purposes of a few examples in this document, we will assume that Theora
 is embedded in an Ogg stream specifically, although this is by no means a
 requirement or fundamental assumption in the Theora design.

The specification for embedding Theora into an Ogg transport stream is given in
 Appendix~\ref{app:oggencapsulation}.

\subsubsection{Codec Setup and Probability Model}

Theora's heritage is the proprietary commerical codec VP3, and it retains a
 fair amount of inflexibility when compared to Vorbis \cite{vorbis}, the first
 Xiph.org codec, which began as a research codec.
However, to provide additional scope for encoder improvement, Theora adopts
 some of the configurable aspects of decoder setup that are present in Vorbis.
This configuration data is not available in VP3, which used hardcoded values
 instead.

Theora makes the same controversial design decision that Vorbis made to include
 the entire probability model for the DCT coefficients and all the quantization
 parameters in the bitstream headers.
This is often several hundred fields.
This makes it impossible to begin decoding at any frame in the stream without
 having previously fetched the codec info and codec setup headers.

\begin{verse}
{\bf Note:} Theora {\em can} initiate decode at an arbitrary intra-frame packet
 within a bitstream so long as the codec has been initialized with the setup
 headers.
\end{verse}

Thus, Theora headers are both required for decode to begin and relatively large
 as bitstream headers go.
The header size is unbounded, although as a rule-of-thumb less than 16kB is
 recommended, and Xiph.org's reference encoder follows this suggestion.
%TODO: Is 8kB enough? My setup header is 7.4kB, that doesn't leave much room
% for comments.
%RG: the lesson from vorbis is that as small as possible is really
% important in some applications. Practically, what's acceptable
% depends a great deal on the target bitrate. I'd leave 16 kB in the
% spec for now. fwiw more than 1k of comments is quite unusual.

Our own design work indicates that the primary liability of the required header
 is in mindshare; it is an unusual design and thus causes some amount of
 complaint among engineers as this runs against current design trends and
 points out limitations in some existing software/interface designs.
However, we find that it does not fundamentally limit Theora's suitable
 application space.

\subsubsection{Format Specification}

The Theora format is well-defined by its decode specification; any encoder that
 produces packets that are correctly decoded by an implementation following
 this specification may be considered a proper Theora encoder.
A decoder must faithfully and completely implement the specification defined
 herein %, except where noted,
 to be considered a proper Theora decoder.
Where appropriate, a non-normative description of encoder processes is
 included.
These sections will be marked as such, and a proper Theora encoder is not
 bound to follow them.

%TODO: \subsubsection{Hardware Profile}

\subsection{Coded Video Structure}

Theora is based on $8\times 8$ blocks of pixels.
This sections describes how a video frame is laid out, divided into blocks, and
 how those blocks are organized.

\subsubsection{Frame Layout}

A video frame in Theora is a two-dimensional array of pixels.
Theora, like VP3, uses a right-handed coordinate system, with the origin in the
 lower-left corner of the frame.
This is contrary to many video formats which use a left-handed coordinate
 system with the origin in the upper-left corner of the frame.
%INT: This means that for interlaced material, the definition of `even fields'
%INT:  and `odd fields' may be reversed between Theora and other video codecs.
%INT: This document will always refer to them as `top fields' and `bottom
%INT:  fields'.

Theora divides the pixel array up into three separate \term{color planes}, one
 for each of the $Y'$, $C_b$, and $C_r$ components of the pixel.
The $Y'$ plane is also called the \term{luma plane}, and the $C_b$ and $C_r$
 planes are also called the \term{chroma planes}.
In some pixel formats, the chroma planes are decimated by two in one or both
 directions.
This means that the width or height of the chroma planes may be half that of
 the total frame width and height, and thus only a multiple of eight, not
 sixteen.
The luma plane is never decimated.

\subsubsection{Picture Region}

A video frame in Theora is required to have a width and height that are
 multiples of sixteen.
However, inside a frame a smaller \term{picture region} may be defined.
The picture region can be offset from the lower-left corner of the frame by up
 to 255 pixels in each direction, and may have an arbitrary width and height,
 provided that it is contained entirely within the coded frame.
It is this picture region that contains the actual video data.
The portions of the frame which lie outside the picture region may contain
 arbitrary data, and should be cropped away after decode.
The picture region plays no other role in the decode process, which operates on
 the entire video frame.

\subsubsection{Blocks and Super Blocks}

Each color plane is subdivided into $8\times 8$ \term{blocks}.
Blocks are grouped into $4\times 4$ arrays called \term{super blocks}.
Each color plane has its own set of blocks and super blocks.
The boundaries of the luma plane are not necessarily aligned with those of the
 chroma planes, if the chroma planes have been decimated.

Blocks are accessed in two different orders in the various decoder processes.
The first is \term{raster order}.
This indexes each block in row-major order, starting in the lower left and
 proceeding along the bottom row, followed by the next row up starting on the
 left, etc.
The second is \term{coded order}.
In coded order, blocks are accessed by super block.
Each super block is traversed in raster order, similar to raster order for
 blocks.
Within each super block, however, blocks are accessed in a Hilbert curve
 pattern, illustrated in Figure~\ref{fig:hilbert-block}.
If a color plane does not contain a complete super block on the top or right
 sides, the same ordering is still used, simply with any blocks outside the
 frame boundary ommitted.

\begin{figure}[htb]
\begin{center}
\setlength{\unitlength}{4144sp}%
%
\begingroup\makeatletter\ifx\SetFigFont\undefined
% extract first six characters in \fmtname
\def\x#1#2#3#4#5#6#7\relax{\def\x{#1#2#3#4#5#6}}%
\expandafter\x\fmtname xxxxxx\relax \def\y{splain}%
\ifx\x\y   % LaTeX or SliTeX?
\gdef\SetFigFont#1#2#3{%
  \ifnum #1<17\tiny\else \ifnum #1<20\small\else
  \ifnum #1<24\normalsize\else \ifnum #1<29\large\else
  \ifnum #1<34\Large\else \ifnum #1<41\LARGE\else
     \huge\fi\fi\fi\fi\fi\fi
  \csname #3\endcsname}%
\else
\gdef\SetFigFont#1#2#3{\begingroup
  \count@#1\relax \ifnum 25<\count@\count@25\fi
  \def\x{\endgroup\@setsize\SetFigFont{#2pt}}%
  \expandafter\x
    \csname \romannumeral\the\count@ pt\expandafter\endcsname
    \csname @\romannumeral\the\count@ pt\endcsname
  \csname #3\endcsname}%
\fi
\fi\endgroup
\begin{picture}(2784,2835)(859,-2821)
\put(901,-2821){\makebox(0,0)[b]{\smash{\SetFigFont{12}{14.4}{rm}{\color[rgb]{0,0,0}0}%
}}}
\put(1801,-2821){\makebox(0,0)[b]{\smash{\SetFigFont{12}{14.4}{rm}{\color[rgb]{0,0,0}1}%
}}}
\put(1801,-1921){\makebox(0,0)[b]{\smash{\SetFigFont{12}{14.4}{rm}{\color[rgb]{0,0,0}2}%
}}}
\put(901,-1921){\makebox(0,0)[b]{\smash{\SetFigFont{12}{14.4}{rm}{\color[rgb]{0,0,0}3}%
}}}
\put(901,-1021){\makebox(0,0)[b]{\smash{\SetFigFont{12}{14.4}{rm}{\color[rgb]{0,0,0}4}%
}}}
\put(901,-121){\makebox(0,0)[b]{\smash{\SetFigFont{12}{14.4}{rm}{\color[rgb]{0,0,0}5}%
}}}
\put(1801,-121){\makebox(0,0)[b]{\smash{\SetFigFont{12}{14.4}{rm}{\color[rgb]{0,0,0}6}%
}}}
\put(1801,-1021){\makebox(0,0)[b]{\smash{\SetFigFont{12}{14.4}{rm}{\color[rgb]{0,0,0}7}%
}}}
\put(2701,-1021){\makebox(0,0)[b]{\smash{\SetFigFont{12}{14.4}{rm}{\color[rgb]{0,0,0}8}%
}}}
\put(2701,-121){\makebox(0,0)[b]{\smash{\SetFigFont{12}{14.4}{rm}{\color[rgb]{0,0,0}9}%
}}}
\put(3601,-121){\makebox(0,0)[b]{\smash{\SetFigFont{12}{14.4}{rm}{\color[rgb]{0,0,0}10}%
}}}
\put(3601,-1021){\makebox(0,0)[b]{\smash{\SetFigFont{12}{14.4}{rm}{\color[rgb]{0,0,0}11}%
}}}
\put(3601,-1921){\makebox(0,0)[b]{\smash{\SetFigFont{12}{14.4}{rm}{\color[rgb]{0,0,0}12}%
}}}
\put(2701,-1921){\makebox(0,0)[b]{\smash{\SetFigFont{12}{14.4}{rm}{\color[rgb]{0,0,0}13}%
}}}
\put(2701,-2821){\makebox(0,0)[b]{\smash{\SetFigFont{12}{14.4}{rm}{\color[rgb]{0,0,0}14}%
}}}
\put(3601,-2821){\makebox(0,0)[b]{\smash{\SetFigFont{12}{14.4}{rm}{\color[rgb]{0,0,0}15}%
}}}
\thinlines
{\color[rgb]{0,0,0}\put(1126,-2761){\vector( 1, 0){450}}
}%
{\color[rgb]{0,0,0}\put(1801,-2536){\vector( 0, 1){450}}
}%
{\color[rgb]{0,0,0}\put(1576,-1861){\vector(-1, 0){450}}
}%
{\color[rgb]{0,0,0}\put(901,-1636){\vector( 0, 1){450}}
}%
{\color[rgb]{0,0,0}\put(901,-736){\vector( 0, 1){450}}
}%
{\color[rgb]{0,0,0}\put(1126,-61){\vector( 1, 0){450}}
}%
{\color[rgb]{0,0,0}\put(1801,-286){\vector( 0,-1){450}}
}%
{\color[rgb]{0,0,0}\put(2026,-961){\vector( 1, 0){450}}
}%
{\color[rgb]{0,0,0}\put(2701,-736){\vector( 0, 1){450}}
}%
{\color[rgb]{0,0,0}\put(2926,-61){\vector( 1, 0){450}}
}%
{\color[rgb]{0,0,0}\put(3601,-286){\vector( 0,-1){450}}
}%
{\color[rgb]{0,0,0}\put(3601,-1186){\vector( 0,-1){450}}
}%
{\color[rgb]{0,0,0}\put(3376,-1861){\vector(-1, 0){450}}
}%
{\color[rgb]{0,0,0}\put(2701,-2086){\vector( 0,-1){450}}
}%
{\color[rgb]{0,0,0}\put(2926,-2761){\vector( 1, 0){450}}
}%
\end{picture}

\end{center}
\caption{Hilbert curve ordering of blocks within a super block}
\label{fig:hilbert-block}
\end{figure}

To illustrate these two orderings, consider a frame that is 240 pixels wide and
 48 pixels high.
Each row of the luma plane has 30 blocks and 8 super blocks, and there are 6
 rows of blocks and one row of super blocks.

When accessed in raster order, each block in the luma plane is assigned the
 following indices:

\vspace{\baselineskip}
\begin{center}
\begin{tabular}{|cccc|c|cc|}\hline
150 & 151 & 152 & 153 & $\ldots$ & 178 & 179 \\
120 & 121 & 122 & 123 & $\ldots$ & 148 & 149 \\\hline
 90 &  91 &  92 &  93 & $\ldots$ & 118 & 119 \\
 60 &  61 &  62 &  63 & $\ldots$ &  88 &  89 \\
 30 &  31 &  32 &  33 & $\ldots$ &  58 &  59 \\
  0 &   1 &   2 &   3 & $\ldots$ &  28 &  29 \\\hline
\end{tabular}
\end{center}
\vspace{\baselineskip}

When accessed in coded order, each block in the luma plane is assigned the
 following indices:

\vspace{\baselineskip}
\begin{center}
\begin{tabular}{|cccc|c|cc|}\hline
123 & 122 & 125 & 124 & $\ldots$ & 179 & 178 \\
120 & 121 & 126 & 127 & $\ldots$ & 176 & 177 \\\hline
  5 &   6 &   9 &  10 & $\ldots$ & 117 & 118 \\
  4 &   7 &   8 &  11 & $\ldots$ & 116 & 119 \\
  3 &   2 &  13 &  12 & $\ldots$ & 115 & 114 \\
  0 &   1 &  14 &  15 & $\ldots$ & 112 & 113 \\\hline
\end{tabular}
\end{center}
\vspace{\baselineskip}

Blocks in the chroma planes immediately follow those of the luma plane without
 a break.

\subsubsection{Macro Blocks}

A macro block contains a $2\times 2$ array of blocks in the luma plane
 {\em and} the co-located blocks in the chroma planes.
Thus macro blocks can represent anywhere from six to twelve blocks, depending
 on how the chroma planes are decimated.
Macro blocks contain information about coding mode and motion vectors for the
 corresponding blocks in all color planes.

Macro blocks are also accessed in a \term{coded order}.
This coded order proceeds be examining each super block in the luma plane in
 raster order, and traversing the four macro blocks inside using a smaller
 Hilbert curve, as shown in Figure~\ref{fig:hilbert-mb}.
If the luma plane does not contain a complete super block on the top or right
 sides, the same ordering is still used, simply with any macro blocks outside
 the frame boundary omitted.
Because the frame size is constrained to be a multiple of 16, there are never
 any partial macro blocks.
Unlike blocks, macro blocks need never be accessed in a pure raster order.

\begin{figure}[htb]
\begin{center}
\setlength{\unitlength}{4144sp}%
%
\begingroup\makeatletter\ifx\SetFigFont\undefined
% extract first six characters in \fmtname
\def\x#1#2#3#4#5#6#7\relax{\def\x{#1#2#3#4#5#6}}%
\expandafter\x\fmtname xxxxxx\relax \def\y{splain}%
\ifx\x\y   % LaTeX or SliTeX?
\gdef\SetFigFont#1#2#3{%
  \ifnum #1<17\tiny\else \ifnum #1<20\small\else
  \ifnum #1<24\normalsize\else \ifnum #1<29\large\else
  \ifnum #1<34\Large\else \ifnum #1<41\LARGE\else
     \huge\fi\fi\fi\fi\fi\fi
  \csname #3\endcsname}%
\else
\gdef\SetFigFont#1#2#3{\begingroup
  \count@#1\relax \ifnum 25<\count@\count@25\fi
  \def\x{\endgroup\@setsize\SetFigFont{#2pt}}%
  \expandafter\x
    \csname \romannumeral\the\count@ pt\expandafter\endcsname
    \csname @\romannumeral\the\count@ pt\endcsname
  \csname #3\endcsname}%
\fi
\fi\endgroup
\begin{picture}(984,1035)(859,-1021)
\put(901,-1021){\makebox(0,0)[b]{\smash{\SetFigFont{12}{14.4}{rm}{\color[rgb]{0,0,0}0}%
}}}
\put(901,-121){\makebox(0,0)[b]{\smash{\SetFigFont{12}{14.4}{rm}{\color[rgb]{0,0,0}1}%
}}}
\put(1801,-121){\makebox(0,0)[b]{\smash{\SetFigFont{12}{14.4}{rm}{\color[rgb]{0,0,0}2}%
}}}
\put(1801,-1021){\makebox(0,0)[b]{\smash{\SetFigFont{12}{14.4}{rm}{\color[rgb]{0,0,0}3}%
}}}
\thinlines
{\color[rgb]{0,0,0}\put(901,-736){\vector( 0, 1){450}}
}%
{\color[rgb]{0,0,0}\put(1126,-61){\vector( 1, 0){450}}
}%
{\color[rgb]{0,0,0}\put(1801,-286){\vector( 0,-1){450}}
}%
\end{picture}

\end{center}
\caption{Hilbert curve ordering of macro blocks within a super block}
\label{fig:hilbert-mb}
\end{figure}

Using the same frame size as the example above, there are 15 macro blocks in
 each row and 3 rows of macro blocks.
They are assigned the following indices:

\vspace{\baselineskip}
\begin{center}
\begin{tabular}{|cc|cc|c|cc|c|}\hline
30 & 31 & 32 & 33 & $\cdots$ & 42 & 43 & 44 \\\hline
 1 &  2 &  5 &  6 & $\cdots$ & 25 & 26 & 29 \\
 0 &  3 &  4 &  7 & $\cdots$ & 24 & 27 & 28 \\\hline
\end{tabular}
\end{center}
\vspace{\baselineskip}

\subsubsection{Predictors}

Each block is coded using one of a small, fixed set of \term{coding modes} that
 define the \term{predictor} for that block's contents.
The INTRA mode uses a constant predictor and is the only mode allowed in intra
 frames.
The other coding modes use the contents of one of two different \term{reference
 frames}.
A reference frame is the fully decoded version of a previous frame in the
 stream.
The first available reference frame is the previous frame, whether it was an
 intra frame or an inter frame.
The second available reference frame is the previous intra frame, called the
 \term{golden frame}.
The most important inter coding mode is INTER\_NOMV, which uses the co-located
 contents of the block in the previous frame as the predictor with no
 motion-compensation, e.g., a motion vector of $(0,0)$.

\subsubsection{DCT Coefficients}

To each block's predictor, a \term{residual} is added to form the final
 contents of the block.
The residual is stored by first applying an integer approximation of a
 two-dimensional Type II Discrete Cosine Transform and then quantizing the
 resulting coefficients.
The DCT takes an an $8\times 8$ array of pixel values as input and returns an
 $8\times 8$ array of coefficient values.
The \term{natural ordering} of these coefficients is defined to be row-major
 order.
They are also often indexed in \term{zig-zag order}, as shown in
 Table~\ref{tab:zig-zag}.

\begin{table}[htb]
\begin{center}
\begin{tabular}[c]{r|c@{}c@{}c@{}c@{}c@{}c@{}c@{}c@{}c@{}c@{}c@{}c@{}c@{}c@{}c}
\multicolumn{1}{r}{} &0&&1&&2&&3&&4&&5&&6&&7 \\\cline{2-16}
0 &  0 &$\rightarrow$&  1 &&  5 &$\rightarrow$&  6 && 14 &$\rightarrow$& 15 && 27 &$\rightarrow$& 28            \\[-0.5\defaultaddspace]
  &    &$\swarrow$&&$\nearrow$& &$\swarrow$&&$\nearrow$& &$\swarrow$&&$\nearrow$& &$\swarrow$&                  \\
1 &  2 &             &  4 &&  7 &             & 13 && 16 &             & 26 && 29 &             & 42            \\[-0.5\defaultaddspace]
  &$\downarrow$&$\nearrow$&&$\swarrow$&&$\nearrow$&&$\swarrow$&&$\nearrow$&&$\swarrow$&&$\nearrow$&$\downarrow$ \\
2 &  3 &             &  8 && 12 &             & 17 && 25 &             & 30 && 41 &             & 43            \\[-0.5\defaultaddspace]
  &    &$\swarrow$&&$\nearrow$& &$\swarrow$&&$\nearrow$& &$\swarrow$&&$\nearrow$& &$\swarrow$&                  \\
3 &  9 &             & 11 && 18 &             & 24 && 31 &             & 40 && 44 &             & 53            \\[-0.5\defaultaddspace]
  &$\downarrow$&$\nearrow$&&$\swarrow$&&$\nearrow$&&$\swarrow$&&$\nearrow$&&$\swarrow$&&$\nearrow$&$\downarrow$ \\
4 & 10 &             & 19 && 23 &             & 32 && 39 &             & 45 && 52 &             & 54            \\[-0.5\defaultaddspace]
  &    &$\swarrow$&&$\nearrow$& &$\swarrow$&&$\nearrow$& &$\swarrow$&&$\nearrow$& &$\swarrow$&                  \\
5 & 20 &             & 22 && 33 &             & 38 && 46 &             & 51 && 55 &             & 60            \\[-0.5\defaultaddspace]
  &$\downarrow$&$\nearrow$&&$\swarrow$&&$\nearrow$&&$\swarrow$&&$\nearrow$&&$\swarrow$&&$\nearrow$&$\downarrow$ \\
6 & 21 &             & 34 && 37 &             & 47 && 50 &             & 56 && 59 &             & 61            \\[-0.5\defaultaddspace]
  &    &$\swarrow$&&$\nearrow$& &$\swarrow$&&$\nearrow$& &$\swarrow$&&$\nearrow$& &$\swarrow$&                  \\
7 & 35 &$\rightarrow$& 36 && 48 &$\rightarrow$& 49 && 57 &$\rightarrow$& 58 && 62 &$\rightarrow$& 63
\end{tabular}
\end{center}
\caption{Zig-zag order}
\label{tab:zig-zag}
\end{table}

Note that the row and column indices refer to {\em frequency number} and not
 pixel locations.
The frequency numbers are defined independently of the memory organization of
 the pixels.
They have been written from top to bottom here to follow conventional notation,
 despite the right-handed coordinate system Theora uses for pixel locations.

Many implementations of the DCT operate `in-place'.
That is, they return DCT coefficients in the same memory buffer that the
 initial pixel values were stored in.
Due to the right-handed coordinate system used for pixel locations in Theora,
 one must note carefully how both pixel values and DCT coefficients are
 organized in memory in such a system.

DCT coefficient $(0,0)$ is called the \term{DC coefficient}.
All the other coefficients are called \term{AC coefficients}.

\subsection{Decoder Configuration}

Decoder setup consists of configuration of the quantization matrices and the
 Huffman codebooks for the DCT coefficients.
The remainder of the decoding pipeline is not configurable.

\subsubsection{Global Configuration}

The global codec configuration consists of a few video related fields, such as
 frame rate, frame size, picture size and offset, aspect ratio, color space,
 pixel format, and a version number.
The version number is divided into a major version, a minor version, amd a
 minor revision number.
For the format defined in this specification, these are `3', `2', and
 `0', respectively, in reference to Theora's origin as a successor to the VP3.2
 format.

\subsubsection{Quantization Matrices}

Theora allows up to 384 different quantization matrices to be defined, one for
 each \term{quantization type} (intra or inter), \term{color plane}
 ($Y'$, $C_b$, or $C_r$), and \term{quantization index}, \qi, which ranges from
 zero to 63, inclusive.
The quantization index generally represents a progressive range of quality
 levels, from low quality near zero to high quality near 63.
However, the interpretation is arbitrary, and it is possible, for example, to
 partition the scale into two completely separate ranges with 32 levels each
 that are meant to represent different classes of source material.

Each quantization matrix is an $8\times 8$ matrix of 16-bit values, which is
 used to quantize the output of the $8\times 8$ DCT.
Quantization matrices are specified using three components: a
 \term{base matrix} and two \term{scale values}.
The first scale value is the \term{DC scale}, which is applied to the DC
 component of the base matrix.
The second scale value is the \term{AC scale}, which is applied to all the
 other components of the base matrix.
There are 64 DC scale values and 64 AC scale values, one for each \qi value.

There are 64 elements in each base matrix, one for each DCT coefficient.
They are stored in natural order.
There is a separate set of base matrices for each quantization type and each
 color plane, with up to 64 possible base matrices in each set, one for each
 \qi value.
Typically the bitstream contains matrices for only a sparse subset of the
 possible \qi values, including at least the first and the last.
The base matrices for the remainder of the \qi values are computed using linear
 interpolation.
This configuration allows the quantization matrices to approximate the complex,
 non-linear processes of the human visual system as the \qi value varies.

Finally, because the in-loop deblocking filter strength depends on the strength
 of the quantization matrices defined in this header, a table of 64 \term{loop
 filter limit values} is defined, one for each \qi value.

The precise specification of how all of this information is decoded appears in
 Section~REF.

\subsubsection{Huffman Codebooks}

Theora uses 80 configurable binary Huffman codes to represent the 32 tokens
 used to encode DCT coefficients.
Each of the 32 token values has a different semantic meaning and is used to
 represent single coefficient values, zero runs, combinations of the two, and
 \term{End-Of-Block markers}.

The 80 codes are divided up into five groups of 16, with each group
 corresponding to a set of DCT coefficient indices.
The first group corresponds to the DC coefficient, while the remaining groups
 correspond to different subsets of the AC coefficients.
Within each frame, two pairs of 4-bit codebook indices are stored.
The first pair selects which codebooks to use from the DC coefficient group for
 the $Y'$ coefficients and the $C_b$ and $C_r$ coefficients.
The second pair selects which codebooks to use from {\em all} of the AC
 coefficient groups for the $Y'$ coefficients and the $C_b$ and $C_r$
 coefficients.

The precise specification of how the codebooks are decoded appears in
 Section~REF.

\subsection{High-Level Decode Process}

\subsubsection{Decoder Setup}

Before decoding can begin, a decoder MUST be initialized using the bitstream
 headers corresponding to the stream to be decoded.
Theora uses three header packets; all are required, in order, by this
 specification.
Once set up, decode may begin at any intra-frame packet---or even inter-frame
 packets, provided the appropriate decoded reference frames have already been
 decoded and cached---belonging to the Theora stream.
In Theora I, all packets after the three initial headers are intra-frame or
 inter-frame packets.

The header packets are, in order, the identification header, the comment
 header, and the setup header.

\paragraph{Identification Header}

The identification header identifies the stream as Theora, provides a version
 number, and defines the characteristics of the video stream such as frame
 size.
A complete description of the identification header appears in
 Section~\ref{sec:idheader}.

\paragraph{Comment Header}

The comment header includes user text comments (`tags') and a vendor string
 for the application/library that produced the stream.
The format of the comment header is the same as that used in the Vorbis I and
 Speex codecs, with slight modifications due to the use of a different bit
 packing mechanism.
A complete description of how the comment header is coded appears in
 Section~\ref{sec:commentheader}, along with a suggested set of tags.

\paragraph{Setup Header}

The setup header includes extensive codec setup information, including the
 complete set of quantization matrices and Huffman codebooks needed to decode
 the DCT coefficients.
A complete description of the setup header appears in Section~REF.

\subsubsection{Decode Procedure}

The decoding and synthesis procedure for all video packets is fundamentally the
 same, with some steps omitted for intra frames.
\begin{enumerate}
\item
Decode packet type flag.
\item
Decode frame header.
\item
Decode coded block information (inter frames only).
\item
Decode macro block mode information (inter frames only).
\item
Decode motion vectors (inter frames only).
\item
Decode block-level \qi information.
\item
Decode DC coefficient for each coded block.
\item
Decode 1st AC coefficient for each coded block.
\item
Decode 2nd AC coefficient for each coded block.
\item
$\ldots$
\item
Decode 63rd AC coefficient for each coded block.
\item Perform DC coefficient prediction.
\item Reconstruct coded blocks.
\item Copy uncoded bocks.
\item Perform loop filtering.
\end{enumerate}

Note that clever rearrangement of the steps in this process is possible.
As an example, in a memory-constrained environment, one can make multiple
 passes through the DCT coefficients to avoid buffering them all in memory.
On the first pass, the starting location of each coefficient is identified, and
 then 64 separate get pointers are used to read in the 64 DCT coefficients
 required to reconstruct each coded block in sequence.
This operation produces entirely equivalent output and is naturally perfectly
 legal.
It may even be a benefit in non-memory-constrained environments due to a
 reduced cache footprint.
The decoder MUST be {\em entirely mathematically equivalent} to the
 specification; it need not be a literal semantic implementation.

Theora makes equivalence easy to check by defining all decoding operations in
 terms of exact integer operations.
No floating-point math is required, and in particular, the implementation of
 the iDCT transform MUST be followed precisely.
This prevents the decoder mismatch problem commonly associated with codecs that
 provide a less rigorous transform specification.
Such a mismatch problem would be devastating to Theora, since a single rounding
 error in one frame could propagate throughout the entire succeeding frame due
 to DC prediction.

\paragraph{Packet Type Decode}

Theora I uses four packet types.
The first three packet types mark each of the three Theora headers described
 above.
The fourth packet type marks a video packet.
All other packet types are reserved; packets marked with a reserved type should
 be ignored.

\paragraph{Frame Header Decode}

The frame header contains some global information about the current frame.
The first is the frame type field, which specifies if this is an intra frame or
 an inter frame.
Inter frames predict their contents from previously decoded reference frames.
Intra frames can be independently decoded with no established reference frames.

The next piece of information in the frame header is the list of \qi values
 allowed in the frame.
Theora allows between one and three different \qi values to be used in a single
 frame, each of which selects a set of six quantization matrices, one for each
 quantization type (inter or intra), and one for each color plane.
The first \qi value is {\em always} used when dequantizing DC coefficients.
The \qi value used when dequantizing AC coefficients, however, can vary from
 block to block.
VP3, in contrast, allowed just a single \qi value per frame for both the DC and
 AC coefficients.

\paragraph{Coded Block Information}

This stage determines which blocks in the frame are coded and which are
 uncoded.
A \term{coded block list} is constructed which lists all the coded blocks in
 coded order.
For intra frames, every block is coded, and so no data needs to be read from
 the packet.

\paragraph{Macro Block Mode Information}

For intra frames, every block is coded in INTRA mode, and this stage can be
 skipped.
In inter frames a \term{coded macro block list} is constructed from the coded
 block list.
Any macro block which has at least one of its luma blocks coded is considered
 coded; all other macro blocks are uncoded, even if they contain coded chroma
 blocks.
A coding mode is decoded for each coded macro block, and assigned to all its
 constituent coded blocks.
All coded chroma blocks in uncoded macro blocks are assigned the INTER\_NOMV
 coding mode.

\paragraph{Motion Vectors}

Intra frames are all coded entirely in INTRA mode, and so this stage can be
 skipped.
Some inter coding modes, however, require one or more motion vectors to be
 specified for each macro block.
These are decoded in this stage, and an appropriate motion vector is assigned
 to each coded block in the macro block.

\paragraph{Block-Level \qi Information}

If a frame allows multiple \qi values, the \qi value assigned to each block is
 decoded here.
Frames that use only a single \qi value have nothing to decode.

\paragraph{DCT Coefficients}

Finally, the quantized DCT coefficients are decoded.
A list of DCT coefficients in zig-zag order for a single block is represented
 by a list of tokens.
A token can take on one of 32 different values, each with a different semantic
 meaning.
A single token can represent a single DCT coefficient, a run of zero
 coefficients within a single block, a combination of a run of zero
 coefficients followed by a single non-zero coefficient, an
 \term{End-Of-Block marker}, or a run of EOB markers.
EOB markers signify that the remainder of the block is one long zero run.
Unlike JPEG and MPEG, each block is not required to end with a special marker.
If non-EOB tokens yield values for all 64 of the coefficients in a block, then
 no EOB marker is needed.

Each token is associated with a specific \term{token index} in a block.
For single-coefficient tokens, this index is the zig-zag index of the token in
 the block.
For zero-run tokens, this index is the zig-zag index of the {\em first}
 coefficient in the run.
For combination tokens, the index is again the zig-zag index of the first
 coefficient in the zero run.
For EOB markers, which signify that the remainder of the block is one long zero
 run, the index is the zig-zag index of the first zero coefficient in that run.
For EOB runs, the token index is that of the first EOB marker in the run.
Due to zero runs and EOB markers, a block does not have to have a token for
 every zig-zag index.

Tokens are grouped in the stream by token index, not by the block they
 originate from.
This means that for each zig-zag index in turn, the tokens with that index from
 {\em all} the coded blocks are coded in coded block order.
When decoding, a current token index is maintained for each coded block.
This index is advanced by the number of coefficients that are added to the
 block as each token is decoded.
After fully decoding all the tokens with token index \ti, the current token
 index of every coded block will be \ti or greater.

If an EOB run of $n$ blocks is decoded at token index \ti, then it ends the
 next $n$ blocks in coded block order whose current token index is equal to
 \ti, but not greater.
If there are fewer than $n$ blocks with a current token index of \ti, then the
 decoder goes through the coded block list again from the start, ending blocks
 with a current token index of $\ti+1$, and so on, until $n$ blocks have been
 ended.

Tokens are read by parsing a Huffman code that depends on \ti and the color
 plane of the next coded block whose current token index is equal to \ti, but
 not greater.
The Huffman codebooks are selected on a per-frame basis from the 80 codebooks
 defined in the setup header.
Many tokens have a fixed number of \term{extra bits} associated with them.
These bits are read from the packet immediately after the token is decoded.
These are used to define things such as coefficient magnitude, sign, and the
 length of runs.

\paragraph{DC Prediction}

After the coefficients for each block are decoded, the quantized DC value of
 each block is adjusted based on the DC values of its neighbors.
This adjustment is performed by scanning the blocks in raster order, not coded
 block order.

\paragraph{Reconstruction}

Finally, using the coding mode, motion vector (if applicable), quantized
 coefficient list, and \qi value defined for each block, all the coded blocks
 are reconstructed.
The DCT coefficients are dequantized, an inverse DCT transform is applied, and
 the predictor is formed from the coding mode and motion vector and added to
 the result.

\paragraph{Loop Filtering}

To complete the reconstructed frame, an in-loop deblocking filter is applied to
 the edges of all coded blocks.

\section{Video Formats}

This section gives a precise description of the video formats that Theora is
 capable of storing.
The Theora bitstream is capable of handling video at any arbitrary resolution
 up to $1048560\times 1048560$.
Such video would require almost three terabytes of storage per frame for
 uncompressed data, so compliant decoders MAY refuse to decode images with
 sizes beyond their capabilities.
%TODO: What MUST a "compliant" decoder accept?
%TODO: What SHOULD a decoder use for an upper bound? (derive from total amount
%TODO:  of memory and memory bandwidth)
%TODO: Any lower limits?
%TODO: We really need hardware device profiles, but such things should be
%TODO:  developed with input from the hardware community.

The remainder of this section talks about two specific aspects of the video
 format: the color space and the pixel format.
The first describes how color is represented and how to transform that color
 representation into a device independent color space such as CIE $XYZ$ (1931).
The second describes the various schemes for sampling the color values in time
 and space.

\subsection{Color Space Conventions}

There are a large number of different color standards used in digital video.
Since Theora is a lossy codec, it restricts itself to only a few of them to
 simplify playback.
Unlike the alternate method of describing all the parameters of the color
 model, this allows a few dedicated routines for color conversion to be written
 and heavily optimized in a decoder.
More flexible conversion functions should instead be specified in an encoder,
 where additional computational complexity is more easily tolerated.
The color spaces were selected to give a fair representation of color standards
 in use around the world today.
Most of the standards that do not exactly match one of these can be converted
 to one fairly easily.

All Theora color spaces are $Y'C_bC_r$ color spaces with one luma channel and
 two chroma channels.
Each channel contains 8-bit discrete values in the range $0\ldots255$, which
 represent non-linear gamma pre-corrected signals.
The Theora identification header contains an 8-bit value that describes the
 color space.
This merely selects one of the color spaces available from an enumerated list.
Currently, only two color spaces are defined, with a third possibility that
 indicates the color space is "unknown".

\subsection{Color Space Conversions and Parameters}
\label{sec:color-xforms}

The parameters which describe the conversions between each color space are
 listed below.
These are the parameters needed to map colors from the encoded $Y'C_bC_r$
 representation to the device-independent color space CIE $XYZ$ (1931).
These parameters define abstract mathematical conversion functions which are
 infinitely precise.
The accuracy and precision with which the conversions are performed in a real
 system is determined by the quality of output desired and the available
 processing power.
Exact decoder output is defined by this specification only in the original
 $Y'C_bC_r$ space.

\begin{description}
\item[$Y'C_bC_r$ to $Y'P_bP_r$:]
\vspace{\baselineskip}\hfill

This conversion takes 8-bit discrete values in the range $[0\ldots255]$ and
 maps them to real values in the range $[0\ldots1]$ for Y and
 $[-\frac{1}{2}\ldots\frac{1}{2}]$ for $P_b$ and $P_r$.
Because some values may fall outside the offset and excursion defined for each
 channel in the $Y'C_bC_r$ space, the results may fall outside these ranges in
 $Y'P_bP_r$ space.
No clamping should be done at this stage.

\begin{eqnarray*}
Y'_\mathrm{out} & = &
 \frac{Y'_\mathrm{in}-\mathrm{Offset}_Y}{\mathrm{Excursion}_Y} \\
P_b             & = &
 \frac{C_b-\mathrm{Offset}_{C_b}}{\mathrm{Excursion}_{C_b}} \\
P_r             & = &
 \frac{C_r-\mathrm{Offset}_{C_r}}{\mathrm{Excursion}_{C_r}}
\end{eqnarray*}

Parameters: $\mathrm{Offset}_{Y,C_b,C_r}$, $\mathrm{Excursion}_{Y,C_b,C_r}$.

\item[$Y'P_bP_r$ to $R'G'B'$:]
\vspace{\baselineskip}\hfill

This conversion takes the one luma and two chroma channel representation and
 maps it to the non-linear $R'G'B'$ space used to drive actual output devices.
Values should be clamped into the range $[0\ldots1]$ after this stage.

\begin{eqnarray*}
R' & = & Y'+2(1-K_r)P_r \\
G' & = & Y'-2\frac{(1-K_b)K_b}{1-K_b-K_r}P_b-2\frac{(1-K_r)K_r}{1-K_b-K_r}P_r\\
B' & = & Y'+2(1-K_b)P_b
\end{eqnarray*}

Parameters: $K_b,K_r$.

\item[$R'G'B'$ to $RGB$ (Output device gamma correction):]
\vspace{\baselineskip}\hfill

This conversion takes the non-linear $R'G'B'$ voltage levels and maps them to
 the linear light levels produced by the actual output device.
Note that this conversion is only that of the output device, and its inverse is
 {\em not} that used by the input device.
Because a dim viewing environment is assumed in most television standards, the
 overall gamma between the input and output devices is usually around $1.1$ to
 $1.2$, and not a strict $1.0$.

For calibration with actual output devices, the model
\begin{displaymath}
L=(E'+\Delta)^\gamma
\end{displaymath}
 should be used, with $\Delta$ the free parameter and $\gamma$ held fixed to
 the value specified in this document.
The conversion function presented here is an idealized version with $\Delta=0$.

\begin{eqnarray*}
R & = & R'^\gamma \\
G & = & G'^\gamma \\
B & = & B'^\gamma
\end{eqnarray*}

Parameters: $\gamma$.

\item[$RGB$ to $R'G'B'$ (Input device gamma correction):]
\vspace{\baselineskip}\hfill

%TODO: Tag section as non-normative

This conversion takes linear light levels and maps them to the non-linear
 voltage levels used to drive the actual input device.
This information is merely informative.
It is not required for building a decoder or for converting between the various
 formats and the actual output capabilities of a particular device.

A linear segment is introduced on the low end to reduce noise in dark areas of
 the image.
The rest of the scale is adjusted so that the power segment of the curve
 intersects the linear segment with the proper slope, and so that it still maps
 0 to 0 and 1 to 1.

\begin{eqnarray*}
R' & = & \left\{
\begin{array}{ll}
\alpha R,                     & 0\le R<\delta   \\
(1+\epsilon)R^\beta-\epsilon, & \delta\le R\le1
\end{array}\right. \\
G' & = & \left\{
\begin{array}{ll}
\alpha G,                     & 0\le G<\delta   \\
(1+\epsilon)G^\beta-\epsilon, & \delta\le G\le1
\end{array}\right. \\
B' & = & \left\{
\begin{array}{ll}
\alpha B,                     & 0\le B<\delta   \\
(1+\epsilon)B^\beta-\epsilon, & \delta\le B\le1
\end{array}\right.
\end{eqnarray*}

Parameters: $\beta$, $\alpha$, $\delta$, $\epsilon$.

\item[$RGB$ to CIE $XYZ$ (1931):]
\vspace{\baselineskip}\hfill

This conversion maps a device-dependent linear RGB space to the
 device-independent linear CIE $XYZ$ space.
The parameters are the CIE chromaticity coordinates of the three
 primaries---red red, green, and blue---as well as the chromaticity coordinates
 of the white point of the device.
This is how hardware manufacturers and standards typically describe a
 particular $RGB$ space.
The math required to convert these parameters into a useful transformation
 matrix is reproduced below.

\begin{eqnarray*}
F                  & = &
\left[\begin{array}{ccc}
\frac{x_r}{y_r}       & \frac{x_g}{y_g}       & \frac{x_b}{y_b}       \\
1                     & 1                     & 1                     \\
\frac{1-x_r-y_r}{y_r} & \frac{1-x_g-y_g}{y_g} & \frac{1-x_b-y_b}{y_b}
\end{array}\right] \\
\left[\begin{array}{c}
s_r \\
s_g \\
s_b
\end{array}\right] & = &
F^{-1}\left[\begin{array}{c}
\frac{x_w}{y_w} \\
1 \\
\frac{1-x_w-y_w}{y_w}
\end{array}\right] \\
\left[\begin{array}{c}
X \\
Y \\
Z
\end{array}\right] & = &
F\left[\begin{array}{c}
s_rR \\
s_gG \\
s_bB
\end{array}\right]
\end{eqnarray*}
Parameters: $x_{r,g,b,q},y_{r,g,b,w}$.

\end{description}

\subsection{Available Color Spaces}
\label{sec:colorspaces}

These are the color spaces currently defined for use by Theora video.
Each one has a short name, with which it is referred to in this document, and
 a more detailed specification of the standards from which its parameters are
 derived.
Some standards do not specify all the parameters necessary.
For these unspecified parameters, this document serves as the definition of
 what should be used when encoding or decoding Theora video.

\subsubsection{Rec. 470M (Rec. ITU-R BT.470-6 System M/NTSC with Rec. ITU-R
 BT.601-5)}
\label{sec:470m}

This color space is used by broadcast television and DVDs in much of the
 Americas, Japan, Korea, and the Union of Myanmar \cite{rec470}.
This color space may also be used for System M/PAL (Brazil), with an
 appropriate conversion supplied by the encoder to compensate for the
 different gamma value.
See Section~\ref{sec:470bg} for an appropriate gamma value to assume for M/PAL
 input.

In the US, studio monitors are adjusted to a D65 white point
 ($x_w,y_w=0.313,0.329$).
In Japan, studio monitors are adjusted to a D white of 9300K
 ($x_w,y_w=0.285,0.293$).

Rec. 470 does not specify a digital encoding of the color signals.
For Theora, Rec. ITU-R BT.601-5 \cite{rec601} is used, starting from the
 $R'G'B'$ signals specified by Rec. 470.

Rec. 470 does not specify an input gamma function.
For Theora, the Rec. 709 \cite{rec709} input function is assumed.
This is the same as that specified by SMPTE 170M \cite{smpte170m}, which claims
 to reflect modern practice in the creation of NTSC signals circa 1994.

The parameters for all the color transformations defined in
 Section~\ref{sec:color-xforms} are given in Table~\ref{tab:470m}.

\begin{table}[htb]
\begin{eqnarray*}
\mathrm{Offset}_{Y,C_b,C_r}      & = & (16, 128, 128)  \\
\mathrm{Excursion}_{Y,C_b,C_r}   & = & (219, 224, 224) \\
K_r                              & = & 0.299           \\
K_b                              & = & 0.114           \\
\gamma                           & = & 2.2             \\
\beta                            & = & 0.45            \\
\alpha                           & = & 4.5             \\
\delta                           & = & 0.018           \\
\epsilon                         & = & 0.099           \\
x_r,y_r                          & = & 0.67, 0.33      \\
x_g,y_g                          & = & 0.21, 0.71      \\
x_b,y_b                          & = & 0.14, 0.08      \\
\mathrm{(Illuminant C)\ }x_w,y_w & = & 0.310, 0.316    \\
\end{eqnarray*}
\caption{Rec. 470M Parameters}
\label{tab:470m}
\end{table}

\subsubsection{Rec. 470BG (Rec. ITU-R BT.470-6 Systems B and G with Rec. ITU-R
 BT.601-5)}
\label{sec:470bg}

This color space is used by the PAL and SECAM systems in much of the rest of
 the world \cite{rec470}
This can be used directly by systems (B, B1, D, D1, G, H, I, K, N)/PAL and (B,
 D, G, H, K, K1, L)/SECAM.

Note that the Rec. 470BG chromaticity values are different from those specified
 in Rec. 470M.
When PAL and SECAM systems were first designed, they were based upon the same
 primaries as NTSC.
However, as methods of making color picture tubes have changed, the primaries
 used have changed as well.
The U.S. recommends using correction circuitry to approximate the existing,
 standard NTSC primaries.
Current PAL and SECAM systems have standardized on primaries in accord with
 more recent technology.

Rec. 470 provisionally permits the use of the NTSC chromaticity values (given
 in Section~\ref{sec:470m}) with legacy PAL and SECAM equipment.
In Theora, material must be decoded assuming the new PAL and SECAM primaries.
Material intended for display on old legacy devices should be converted by the
 decoder.

The official Rec. 470BG specifies a gamma value of $\gamma=2.8$.
However, in practice this value is unrealistically high \cite{Poyn97}.
Rec. 470BG states that the overall system gamma should be approximately
 $\gamma\beta=1.2$.
Since most cameras pre-correct with a gamma value of $\beta=0.45$,
 this suggests an output device gamma of approximately $\gamma=2.67$.
This is the value recommended for use with PAL systems in Theora.

Rec. 470 does not specify a digital encoding of the color signals.
For Theora, Rec. ITU-R BT.601-5 \cite{rec601} is used, starting from the
 $R'G'B'$ signals specified by Rec. 470.

Rec. 470 does not specify an input gamma function.
For Theora, the Rec 709 \cite{rec709} input function is assumed.

The parameters for all the color transformations defined in
 Section~\ref{sec:color-xforms} are given in Table~\ref{tab:470bg}.

\begin{table}[htb]
\begin{eqnarray*}
\mathrm{Offset}_{Y,C_b,C_r}    & = & (16, 128, 128)  \\
\mathrm{Excursion}_{Y,C_b,C_r} & = & (219, 224, 224) \\
K_r                            & = & 0.299           \\
K_b                            & = & 0.114           \\
\gamma                         & = & 2.67            \\
\beta                          & = & 0.45            \\
\alpha                         & = & 4.5             \\
\delta                         & = & 0.018           \\
\epsilon                       & = & 0.099           \\
x_r,y_r                        & = & 0.64, 0.33      \\
x_g,y_g                        & = & 0.29, 0.60      \\
x_b,y_b                        & = & 0.15, 0.06      \\
\mathrm{(D65)\ }x_w,y_w        & = & 0.313, 0.329    \\
\end{eqnarray*}
\caption{Rec. 470BG Parameters}
\label{tab:470bg}
\end{table}

\subsection{Pixel Formats}

\section{Bitpacking Convention}
\label{sec:bitpacking}

\subsection{Overview}

The Theora codec uses relatively unstructured raw packets containing
 binary integer fields of arbitrary width.
Logically, each packet is a bitstream in which bits are written one-by-one by
 the encoder and then read one-by-one in the same order by the decoder.
Most current binary storage arrangements group bits into a native storage unit
 of eight bits (octets), sixteen bits, thirty-two bits, or less commonly other
 fixed sizes.
The Theora bitpacking convention specifies the correct mapping of the logical
 packet bitstream into an actual representation in fixed-width units.

\subsubsection{Octets and Bytes}

In most contemporary architectures, a `byte' is synonymous with an `octect',
 that is, eight bits.
For purposes of the bitpacking convention, a byte implies the smallest native
 integer storage representation offered by a platform.
Modern file systems invariably offer bytes as the fundamental atom of storage.

The most ubiquitous architectures today consider a `byte' to be an octet.
Note, however, that the Theora bitpacking convention is still well defined for
 any native byte size; an implementation can use the native bit-width of a
 given storage system.
This document assumes that a byte is one octet for purposes of example only.

\subsubsection{Words and Byte Order}

A `word' is an integer size that is a grouped multiple of the byte size.
Most architectures consider a word to be a group of two, four, or eight bytes.
Each byte in the word can be ranked by order of `significance', e.g. the
 significance of the bits in each byte when storing a binary integer in the
 word.
Several byte orderings are possible in a word.
The common ones are
\begin{itemize}
\item{Big-endian:}
in which the most significant byte comes first, e.g. 3-2-1-0,
\item{Little-endian:}
in which the least significant byte comes first, e.g. 0-1-2-3, and
\item{Mixed-endian:}
one of the less-common orderings that cannot be put into the above two
 categories, e.g. 3-1-2-0 or 0-2-1-3.
\end{itemize}

The Theora bitpacking convention specifies storage and bitstream manipulation
 at the byte, not word, level.
Thus host word ordering is of a concern only during optimization, when writing
 code that operates on a word of storage at a time rather than a byte.
Logically, bytes are always encoded and decoded in order from byte zero through
 byte $n$.

\subsubsection{Bit Order}

A byte has a well-defined `least significant' bit (LSb), which is the only bit
 set when the byte is storing the two's complement integer value $+1$.
A byte's `most significant' bit (MSb) is at the opposite end.
Bits in a byte are numbered from zero at the LSb to $n$ for the MSb, where
 $n=7$ in an octet.

\subsection{Coding Bits into Bytes}

The Theora codec needs to encode arbitrary bit-width integers from zero to 32
 bits wide into packets.
These integer fields are not aligned to the boundaries of the byte
 representation; the next field is read at the bit position immediately
 after the end of the previous field.

The decoder logically unpacks integers by first reading the MSb of a binary
 integer from the logical bitstream, followed by the next most significant
 bit, etc., until the required number of bits have been read.
When unpacking the bytes into bits, the decoder begins by reading the MSb of
 the integer to be read from the most significant unread bit position of the
 source byte, followed by the next-most significant bit position of the
 destination integer, and so on up to the requested number of bits.
Note that this differs from the Vorbis I codec, which
 begins decoding with the LSb of the source integer, reading it from the
 LSb of the source byte.
When all the bits of the current source byte are read, decoding continues with
 the MSb of the next byte.
Any unfilled bits in the last byte of the packet MUST be cleared to zero by the
 encoder.

\subsubsection{Signedness}

The binary integers decoded by the above process may be either signed or
 unsigned.
This varies from integer to integer, and this specification
 indicates how each value should be interpreted as it is read.
That is, depending on context, the three bit binary pattern `b111' can be taken
 to represent either `$7$' as an unsigned integer or `$-1$' as a signed, two's
 complement integer.

\subsubsection{Encoding Example}

The following example shows the state of an (8-bit) byte stream after several
 binary integers are encoded, including the location of the put pointer for the
 next bit to write to and the total length of the stream in bytes.

Encode the 4 bit unsigned integer value `12' (b1100) into an empty byte stream.

\begin{tabular}{r|ccccccccl}
\multicolumn{1}{r}{}& &&&&$\downarrow$&&&& \\
         & 7 & 6 & 5 & 4 & 3 & 2 & 1 & 0 & \\\cline{1-9}
byte 0   & \textbf{1} & \textbf{1} & \textbf{0} & \textbf{0} &
                           0 & 0 & 0 & 0 & $\leftarrow$     \\
byte 1   & 0 & 0 & 0 & 0 & 0 & 0 & 0 & 0 &                  \\
byte 2   & 0 & 0 & 0 & 0 & 0 & 0 & 0 & 0 &                  \\
byte 3   & 0 & 0 & 0 & 0 & 0 & 0 & 0 & 0 &                  \\
\multicolumn{1}{c|}{$\vdots$}&\multicolumn{8}{c}{$\vdots$}& \\
byte $n$ & 0 & 0 & 0 & 0 & 0 & 0 & 0 & 0 &
byte stream length: 1 byte
\end{tabular}
\vspace{\baselineskip}

Continue by encoding the 3 bit signed integer value `-1' (b111).

\begin{tabular}{r|ccccccccl}
\multicolumn{1}{r}{} &&&&&&&&$\downarrow$& \\
         & 7 & 6 & 5 & 4 & 3 & 2 & 1 & 0 & \\\cline{1-9}
byte 0   & \textbf{1} & \textbf{1} & \textbf{0} & \textbf{0} &
           \textbf{1} & \textbf{1} & \textbf{1} & 0 & $\leftarrow$ \\
byte 1   & 0 & 0 & 0 & 0 & 0 & 0 & 0 & 0 &                         \\
byte 2   & 0 & 0 & 0 & 0 & 0 & 0 & 0 & 0 &                         \\
byte 3   & 0 & 0 & 0 & 0 & 0 & 0 & 0 & 0 &                         \\
\multicolumn{1}{c|}{$\vdots$}&\multicolumn{8}{c}{$\vdots$}&        \\
byte $n$ & 0 & 0 & 0 & 0 & 0 & 0 & 0 & 0 &
byte stream length: 1 byte
\end{tabular}
\vspace{\baselineskip}

Continue by encoding the 7 bit integer value `17' (b0010001).

\begin{tabular}{r|ccccccccl}
\multicolumn{1}{r}{} &&&&&&&$\downarrow$&& \\
         & 7 & 6 & 5 & 4 & 3 & 2 & 1 & 0 & \\\cline{1-9}
byte 0   & \textbf{1} & \textbf{1} & \textbf{0} & \textbf{0} &
           \textbf{1} & \textbf{1} & \textbf{1} & \textbf{0} & \\
byte 1   & \textbf{0} & \textbf{1} & \textbf{0} & \textbf{0} &
           \textbf{0} & \textbf{1} & 0 & 0 & $\leftarrow$      \\
byte 2   & 0 & 0 & 0 & 0 & 0 & 0 & 0 & 0 &                     \\
byte 3   & 0 & 0 & 0 & 0 & 0 & 0 & 0 & 0 &                     \\
\multicolumn{1}{c|}{$\vdots$}&\multicolumn{8}{c}{$\vdots$}&    \\
byte $n$ & 0 & 0 & 0 & 0 & 0 & 0 & 0 & 0 &
byte stream length: 2 bytes
\end{tabular}
\vspace{\baselineskip}

Continue by encoding the 13 bit integer value `6969' (b11011 00111001).

\begin{tabular}{r|ccccccccl}
\multicolumn{1}{r}{} &&&&$\downarrow$&&&&& \\
         & 7 & 6 & 5 & 4 & 3 & 2 & 1 & 0 &            \\\cline{1-9}
byte 0   & \textbf{1} & \textbf{1} & \textbf{0} & \textbf{0} &
           \textbf{1} & \textbf{1} & \textbf{1} & \textbf{0} & \\
byte 1   & \textbf{0} & \textbf{1} & \textbf{0} & \textbf{0} &
           \textbf{0} & \textbf{1} & \textbf{1} & \textbf{1} & \\
byte 2   & \textbf{0} & \textbf{1} & \textbf{1} & \textbf{0} &
           \textbf{0} & \textbf{1} & \textbf{1} & \textbf{1} & \\
byte 3   & \textbf{0} & \textbf{0} & \textbf{1} &
                       0 & 0 & 0 & 0 & 0 & $\leftarrow$        \\
\multicolumn{1}{c|}{$\vdots$}&\multicolumn{8}{c}{$\vdots$}&    \\
byte $n$ & 0 & 0 & 0 & 0 & 0 & 0 & 0 & 0 &
byte stream length: 4 bytes
\end{tabular}
\vspace{\baselineskip}

\subsubsection{Decoding Example}

The following example shows the state of the (8-bit) byte stream encoded in the
 previous example after several binary integers are decoded, including the
 location of the get pointer for the next bit to read.

Read a two bit unsigned integer from the example encoded above.

\begin{tabular}{r|ccccccccl}
\multicolumn{1}{r}{} &&&$\downarrow$&&&&&&              \\
         & 7 & 6 & 5 & 4 & 3 & 2 & 1 & 0 &              \\\cline{1-9}
byte 0   & \textbf{1} & \textbf{1} & 0 & 0 & 1 & 1 & 1 & 0 & $\leftarrow$ \\
byte 1   & 0 & 1 & 0 & 0 & 0 & 1 & 1 & 1 &              \\
byte 2   & 0 & 1 & 1 & 0 & 0 & 1 & 1 & 1 &              \\
byte 3   & 0 & 0 & 1 & 0 & 0 & 0 & 0 & 0 &
byte stream length: 4 bytes
\end{tabular}
\vspace{\baselineskip}

Value read: 3 (b11).

Read another two bit unsigned integer from the example encoded above.

\begin{tabular}{r|ccccccccl}
\multicolumn{1}{r}{} &&&&&$\downarrow$&&&&              \\
         & 7 & 6 & 5 & 4 & 3 & 2 & 1 & 0 &              \\\cline{1-9}
byte 0   & \textbf{1} & \textbf{1} & \textbf{0} & \textbf{0} &
                           1 & 1 & 1 & 0 & $\leftarrow$ \\
byte 1   & 0 & 1 & 0 & 0 & 0 & 1 & 1 & 1 &              \\
byte 2   & 0 & 1 & 1 & 0 & 0 & 1 & 1 & 1 &              \\
byte 3   & 0 & 0 & 1 & 0 & 0 & 0 & 0 & 0 &
byte stream length: 4 bytes
\end{tabular}
\vspace{\baselineskip}

Value read: 0 (b00).

Two things are worth noting here.
\begin{itemize}
\item
Although these four bits were originally written as a single four-bit integer,
 reading some other combination of bit-widths from the bitstream is well
 defined.
No artificial alignment boundaries are maintained in the bitstream.
\item
The first value is the integer `$3$' only because the context stated we were
 reading an unsigned integer.
Had the context stated we were reading a signed integer, the returned value
 would have been the integer `$-1$'.
\end{itemize}

\subsubsection{End-of-Packet Alignment}

The typical use of bitpacking is to produce many independent byte-aligned
 packets which are embedded into a larger byte-aligned container structure,
 such as an Ogg transport bitstream.
Externally, each bitstream encoded as a byte stream MUST begin and end on a
 byte boundary.
Often, the encoded packet bitstream is not an integer number of bytes, and so
 there is unused space in the last byte of a packet.

Unused space in the last byte of a packet is always zeroed during the encoding
 process.
Thus, should this unused space be read, it will return binary zeroes.
There is no marker pattern or stuffing bits that will allow the decoder to
 obtain the exact size, in bits, of the original bitstream.
This knowledge is not required for decoding.

Attempting to read past the end of an encoded packet results in an
 `end-of-packet' condition.
Any further read operations after an `end-of-packet' condition shall also
 return `end-of-packet'.
Unlike Vorbis, Theora does not use truncated packets as a normal mode of
 operation.
Therefore if a decoder encounters the `end-of-packet' condition during normal
 decoding, it may attempt to use the bits that were read to recover as much of
 encoded data as possible, signal a warning or error, or both.

\subsubsection{Reading Zero Bit Integers}

Reading a zero bit integer returns the value `$0$' and does not increment
 the stream pointer.
Reading to the end of the packet, but not past the end, so that an
 `end-of-packet' condition is not triggered, and then reading a zero bit
 integer shall succeed, returning `$0$', and not trigger an `end-of-packet'
 condition.
Reading a zero bit integer after a previous read sets the `end-of-packet'
 condition shall fail, also returning `end-of-packet'.

\section{Bitstream Headers}
\label{sec:headers}

A Theora bitstream begins with three header packets.
The header packets are, in order, the identification header, the comment
 header, and the setup header.
All are required for decode compliance.
An end-of-packet condition encountered while decoding the identification or
 setup header packets renders the stream undecodable.
An end-of-packet condition encountered while decode the comment header is a
 non-fatal error condition, and MAY be ignored by a decoder.

\subsection{Common Header Decode}

Each header packet begins with the same header fields:

\begin{enumerate}
\item{\bitvar{packet\_type}:} 8 bit unsigned integer.
\item{0x74, 0x68, 0x65, 0x6F, 0x72, 0x61:}
The characters `t', `h', `e', `o', `r', and `a' as 8 bit unsigned integers.
\end{enumerate}

Decode continues according to packet type.
The identification header is type 0x80, the comment header is type 0x81, and
 the setup header is type 0x82.
These types all have their high bit set, as a packet with its first bit unset
 is a video data packet.
These packets must occur in the order: identification, comment, setup.

\subsection{Identification Header}
\label{sec:idheader}

The identification header is a short header with only a few fields used to
 declare the stream definitively as Theora and provide detailed information
 about the format of the fully decoded video data.
The identification header is coded as follows:

\begin{enumerate}
\item{\bitvar{version\_major}:} 8-bit unsigned integer.
\item{\bitvar{version\_minor}:} 8-bit unsigned integer.
\item{\bitvar{version\_revision}:} 8-bit unsigned integer.
\item{\bitvar{frame\_mb\_width}:} 16-bit unsigned integer.
\item{\bitvar{frame\_mb\_height}:} 16-bit unsigned integer.
\item{\bitvar{picture\_width}:} 24-bit unsigned integer.
\item{\bitvar{picture\_height}:} 24-bit unsigned integer.
\item{\bitvar{picture\_x\_offset}:} 8-bit unsigned integer.
\item{\bitvar{picture\_y\_offset}:} 8-bit unsigned integer.
\item{\bitvar{frame\_rate\_numerator}:} 32-bit unsigned integer.
\item{\bitvar{frame\_rate\_denominator}:} 32-bit unsigned integer.
\item{\bitvar{pixel\_aspect\_numerator}:} 24-bit unsigned integer.
\item{\bitvar{pixel\_aspect\_denominator}:} 24-bit unsigned integer.
\item{\bitvar{color\_space}:} 8-bit unsigned integer.
\item{\bitvar{nominal\_bitrate}:} 24-bit unsigned integer.
\item{\bitvar{quality}:} 6-bit unsigned integer.
\item{\bitvar{keyframe\_granule\_shift}:} 5-bit unsigned integer.
\item{\bitvar{pixel\_format}:} 2-bit unsigned integer.
\item{\bitvar{reserved}:} 3-bit unsigned integer.
\end{enumerate}

\bitvar{version\_major}, \bitvar{version\_minor}, and
 \bitvar{version\_revision} MUST be $3$, $2$, and $0$, respectively in order
 to be compatible with this document.

Both \bitvar{frame\_mb\_width} and \bitvar{frame\_mb\_height} MUST be greater
 than zero.
Each specifies the width of the coded video frame in macro blocks.
The actual width of the frame in pixels is $16*\bitvar{frame\_mb\_width}$, and
 the height in pixels is $16*\bitvar{frame\_mb\_height}$.
The size of the displayable picture within this coded frame in pixels is
 \bitvar{picture\_width} by \bitvar{picture\_height}.
The lower-left corner of the displayable picture is located in position
 $(\bitvar{picture\_x\_offset},$ $\bitvar{picture\_y\_offset})$.
These MUST be less than the frame width and frame height in pixels,
 respectively.
In addition, $\bitvar{picture\_x\_offset}+\bitvar{picture\_width}$ and
 $\bitvar{picture\_y\_offset}+\bitvar{picture\_height}$ MUST be less than the
 frame width and frame height in pixels, respectively.

If any of these checks fail, the stream is rendered undecodable.

Theora is a fixed-frame rate video codec.
Frames are sampled at the constant rate of
 $\frac{\bitvar{frame\_rate\_numerator}}{\bitvar{frame\_rate\_denominator}}$
 frames per second.
Both of these fields MUST be greater than zero, or the stream is rendered
 undecodable.

The aspect ratio of the pixels within a frame, defined as the ratio of the
 physical width of the pixel to its physical height, is specified by the ratio
 $\bitvar{pixel\_aspect\_numerator}:\bitvar{pixel\_aspect\_denominator}$.
Either of these fields MAY be zero, in which case the pixel aspect ratio
 defaults to $1:1$.

The \bitvar{nominal\_bitrate} field is used only as a hint.
For pure VBR streams, this value may be considerably off.
The field MAY be set to zero to indicate that the encoder did not care to
 speculate.
%TODO: Quality values... this is also a hint, but of what?
%TODO: ideally, it should be semantically distinct from the \qi values.

The \bitvar{keyframe\_granule\_shift} is used to partition the granule
 position associated with each packet into two different parts.
The frame number of the last keyframe, starting from zero, is stored in the
 upper $64-\bitvar{keyframe\_granule\_shift}$ bits, while the lower
 \bitvar{keyframe\_granule\_shift} bits contain the number of frames since the
 last keyframe.
Complete details on the granule position mapping are specified in Section~REF.

The \bitvar{color\_space} field contains a value from an enumerated list of
 the available color spaces, given in Table~\ref{tab:colorspaces}.
The `Undefined' value indicates that color space information was not
 available to the encoder.
It MAY be specified by the application via an external means.
If a reserved value is given, a decoder MAY refuse to decode the stream.

\begin{table}[htb]
\begin{center}
\begin{tabular*}{215pt}{cl@{\extracolsep{\fill}}}\toprule
Value        & Color Space                               \\\midrule
$0$          & Undefined.                                \\
$1$          & Rec. 470M (see Section~\ref{sec:470m}).   \\
$2$          & Rec. 470BG (see Section~\ref{sec:470bg}). \\
$3\ldots255$ & Reserved.                                 \\\bottomrule
\end{tabular*}
\end{center}
\caption{Enumerated List of Color Spaces}
\label{tab:colorspaces}
\end{table}

The \bitvar{pixel\_format} field contains a value from an enumerated list of
 the available pixel formats, given in Table~\ref{tab:pixel-formats}.
If the reserved value $1$ is given, the stream is rendered undecodable.

\begin{table}[htb]
\begin{center}
\begin{tabular*}{215pt}{cl@{\extracolsep{\fill}}}\toprule
Value & Pixel Format             \\\midrule
$0$   & 4:2:0 (see Section~REF). \\
$1$   & Reserved.                \\
$2$   & 4:2:2 (see Section~REF). \\
$3$   & 4:4:4 (see Section~REF). \\\bottomrule
\end{tabular*}
\end{center}
\caption{Enumerated List of Pixel Formats}
\label{tab:pixel-formats}
\end{table}

Finally, the bits in the \bitvar{reserved} field MUST be zero, or the stream
 is rendered undecodable.

\subsection{Comment Header}
\label{sec:commentheader}

The Theora comment header is the second of three header packets that begin a
 Theora stream.
It is meant for short text comments, not aribtrary metadata; arbitrary metadata
 belongs in a separate logical stream that provides greater structure and
 machine parseability.

The comment field is meant to be used much like someone jotting a quick note on
 the bottom of a CDR.
It should be a little information to remember the disc by and explain it to
 others; a short, to-the-point text note taht need not only be a couple words,
 but isn't going to be more than a short paragraph.
The essentials, in other words, whatever they turn out to be, e.g.:

%TODO: Example

\subsubsection{Comment Header Coding}

The comment header is stored as a logical list of eight-bit clean vectors; the
 number of vectors is bounded at $2^{32}-1$ and the length of each vector is
 limited to $2^{32}-1$ bytes.
The vector length is encoded; the vector contents themselves are not null
 terminated.
In addition to the vector list, there is a single vector for a vendor name,
 also eight-bit clean with a length encoded in 32 bits.
%TODO: The 1.0 release of libtheora sets the vendor string to ...

The comment header is decoded as follows:
\begin{enumerate}
\item{\bitvar{vendor\_length\_0}:} 8-bit unsigned integer.
\item{\bitvar{vendor\_length\_1}:} 8-bit unsigned integer.
\item{\bitvar{vendor\_length\_2}:} 8-bit unsigned integer.
\item{\bitvar{vendor\_length\_3}:} 8-bit unsigned integer.
\item{\bitvar{vendor\_string}:} \bitvar{vendor\_length} 8-bit unsigned
 integers.
\item{\bitvar{user\_comment\_list\_length\_0}:} 8-bit unsigned integer.
\item{\bitvar{user\_comment\_list\_length\_1}:} 8-bit unsigned integer.
\item{\bitvar{user\_comment\_list\_length\_2}:} 8-bit unsigned integer.
\item{\bitvar{user\_comment\_list\_length\_3}:} 8-bit unsigned integer.
\item{\bitvar{user\_comment\_list}:} \bitvar{user\_comment\_list\_length}
 user comments.
\end{enumerate}

Here \bitvar{vendor\_length} and \bitvar{user\_comment\_list\_length} are
 formed by arranging their constituent octets in little-endian order.
\begin{eqnarray*}
\bitvar{vendor\_length} & = &
\bitvar{vendor\_length\_0} + \\
&& \bitvar{vendor\_length\_1}*2^8 + \\
&& \bitvar{vendor\_length\_2}*2^{16} + \\
&& \bitvar{vendor\_length\_3}*2^{32} \\
\bitvar{user\_comment\_list\_length} & = &
\bitvar{user\_comment\_list\_length\_0} + \\
&& \bitvar{user\_comment\_list\_length\_1}*2^8 + \\
&& \bitvar{user\_comment\_list\_length\_2}*2^{16} + \\
&& \bitvar{user\_comment\_list\_length\_3}*2^{32}
\end{eqnarray*}
This construction is used so that on platforms with 8-bit bytes, the memory
 organization of the comment header is identical with that of Vorbis I,
 allowing for common parsing code despite the different bit packing
 conventions.

Each user comment is similarly decoded as:
\begin{enumerate}
\item{$\bitvar{comment\_length\_0}[i]$:} 8-bit unsigned integer.
\item{$\bitvar{comment\_length\_1}[i]$:} 8-bit unsigned integer.
\item{$\bitvar{comment\_length\_2}[i]$:} 8-bit unsigned integer.
\item{$\bitvar{comment\_length\_3}[i]$:} 8-bit unsigned integer.
\item{$\bitvar{comment\_string}[i]$:} $\bitvar{comment\_length}[i]$ 8-bit
 unsigned integers.
\end{enumerate}

Again, $\bitvar{comment\_length}[i]$ is formed as follows:
\begin{eqnarray*}
\bitvar{comment\_length}[i] & = &
\bitvar{comment\_length\_0}[i] + \\
&& \bitvar{comment\_length\_1}[i]*2^8 + \\
&& \bitvar{comment\_length\_2}[i]*2^{16} + \\
&& \bitvar{comment\_length\_3}[i]*2^{32} \\
\end{eqnarray*}

The comment header comprises the entirety of the second header packet.
Unlike the first header packet, it is not generally the only packet on the
 second page and may span multiple pages.
The length of the comment header packet is (practically) unbounded.
The comment header packet is not optional; it must be present in the stream
 even if it is logically empty.

\subsubsection{User Comment Format}

The user comment vectors are structured similarly to a UNIX environment
 variable.
That is, comment fields consist of a field name and a corresponding value and
 look like:
\begin{center}
\begin{tabular}{rcl}
$\bitvar{comment\_string}[0]$ & = & ``TITLE=the look of Theora" \\
$\bitvar{comment\_string}[1]$ & = & ``DIRECTOR=me"
\end{tabular}
\end{center}

The field name is case-insensitive and MUST consist of ASCII characters 0x20
 through 0x7D, 0x3D (`=') excluded.
ASCII 0x41 through 0x5A inclusive (characters `A'--`Z') are to be considered
 equivalent to ASCII 0x61 through 0x7A inclusive (characters `a'--`z').
%TODO: Is an empty field-name permitted?

The field name is immediately followed by ASCII 0x3D (`='); this equals sign is
 used to terminate the field name.

The data immediately after 0x3D until the end of the vector is the eight-bit
 clean value of the field contents encoded as a UTF-8 string.
%TODO: Cite UTF-8 standard.

Field names MUST not be `internationalized'; this is a concession to
 simplicity, not an attempt to exclude the majority of the world that doesn't
 speak English.
Applications MAY wish to present internationalized versions of the standard
 field names listed below to the user, but they are not to be stored in the
 bitstream.
Field {\em contents}, however, use the UTF-8 character encoding to allow easy
 representation of any language.

Individual `vendors' MAY use non-standard field names within reason.
The proper use of comment fields as human-readable notes has already been
 explained.
Abuse will be discouraged.

There is no vendor-specific prefix to `non-standard' field names.
Vendors SHOULD make some effort to avoid arbitrarily polluting the common
 namespace.
We will generally collect the more useful tags here to help with
 standardization.

Field names are not restricted to occur only once within a comment header.
%TODO: Example

\paragraph{Field Names}

Below is a proposed, minimal list of standard field names with a description of
 their intended use.
No field names are mandatory; a comment header may contain one or more, all, or
 none of the names in this list.

\begin{description}
\item{TITLE:} Video name.
%TODO: Complete list
\end{description}

\appendix

\section{Ogg Bitstream Encapsulation}
\label{app:oggencapsulation}

\subsection{Overview}

This document specifies the embedding or encapsulation of Theora packets
 in an Ogg transport stream.

Ogg is a stream oriented wrapper for coded, linear time-based data.
It provides syncronization, multiplexing, framing, error detection and
 seeking landmarks for the decoder and complements the raw packet format
 used by the Theora codec.

This document assumes familiarity with the details of the Ogg standard.
The Xiph.org documentation provides an overview of the Ogg transport stream
 format \cite{oggstream} and a detailed description \cite{oggframe}.
%TODO: Maybe we should just put these links in-line, instead of as references.
The format is also defined in RFC~3533 \cite{rfc3533}.
While Theora packets can be embedded in a wide variety of media
 containers and streaming mechanisms, the Xiph.org Foundation
 recommends Ogg as the native format for Theora video in file-oriented
 storage and transmission contexts.

\subsubsection{MIME type}

The correct MIME type of any Ogg file is {\tt application/ogg}.
Outside of an encapsulation, the mime type {\tt video/x-theora} may
 be used to refer specifically to the Theora compressed video stream.

\subsection{Embedding in a logical bitstream}

Ogg separates a {\em logical bitstream} consisting of the framing of
 a particular sequence of packets and complete within itself from
 the {\em physical bitstream} which may consist either of a single
 logical bitstream or a number of logical bitstreams multiplexed
 together.
This section specifies the embedding of Theora packets in a logical Ogg
 bitstream.
The mapping of Ogg Theora logical bitstreams into a multiplexed physical Ogg
 stream is described in the next section.

\subsubsection{Headers}

The initial info header packet appears by itself in a single Ogg page.
This page defines the start of the logical stream and MUST have
 the `beginning of stream' flag set.

The second and third header packets (metadata comments and decoder
 setup data) can together span one or more Ogg pages.
If there are additional non-normative header packets, they MUST be
 included in this sequence of pages as well.
The comment header packet MUST begin the second Ogg page in the logical
 bitstream, and there MUST be a page break between the last header
 packet and the first frame data packet.

These two page break requirements facilitate stream identification and
 simplify header acquisition for seeking and live streaming applications.

All header pages MUST have their granule position field set to zero.
%TODO: or -1?
%TBT: What are we doing now?

\subsubsection{Frame data}

The first frame data packet in a logical bitstream MUST begin a fresh page.
All other data packets are placed one at a time into Ogg pages
 until the end of the stream.
Packets can span pages and multiple packets can be placed within any
 one page.
The last page in the logical bitstream MUST have its `end of stream'
 flag set.

Frame data pages MUST be marked with a granule index corresponding to
 the display time of the last frame/packet that finishes in that page.

{\bf Note:}
This scheme is still under discussion.
It has also been proposed that pages be labeled with a granule corresponding to
 the first frame that begins on that page.
This simplifies seeking and mux, but is different from the published
 definition of the Ogg granule field.
This document will be updated when the issue is settled.

%TODO: \subsubsection{Granule position}

\subsection{Multiplexed stream mapping}

Applications supporting Ogg Theora I must support Theora bitstreams
 multiplexed with compressed audio data in the Vorbis I and Speex
 formats, and should support Ogg-encapsulated MNG graphics for overlays.
% and the Writ format for text-based titling.
%TBT: That's great... do these things have specifications?

Multiple audio and video bitstreams may be multiplexed together.
How playback of multiple/alternate streams is handled is up to the
 application.
Some conventions based on included metadata aide interoperability
 in this respect.
%TODO: describe multiple vs. alternate streams, language mapping
% and reference metadata descriptions.

\subsubsection{Chained streams}

Ogg Theora decoders and playback applications MUST support both grouped
 streams (multiplexed concurrent logical streams) and chained streams
 (sequential concatenation of independent physical bitstreams).

The number and codec data types of multiplexed streams and the decoder
 parameters for those stream types that re-occur can all change at a
 chaining boundary.
A playback application MUST be prepared to handle such changes and
 SHOULD do so smoothly with the minimum possible visible disruption.
The specification of grouped streams below applies independently to each
 segment of a chained bitstream.

\subsubsection{Grouped streams}

At the beginning of a multiplexed stream, the `beginning of stream'
 pages for each logical bitstream will be grouped together.
Within these, the first page to occur MUST be the Theora page.
This facilitates identification of Ogg Theora files among other
 Ogg-encapsulated content.
A playback application must nevertheless handle streams where this
 arrangement is not correct.
%TBT: Then what's the point of requiring it in the spec?

If there is more than one Theora logical stream, the first page should
 be from the primary stream.
That is, the best choice for the stream a generic player should begin
 displaying without special user direction.
If there is more than one audio stream, or of any other stream
 type, the identification page of the primary stream of that type
 should be placed before the others.
%TBT: That's all pretty vague.

After the `beginning of stream' pages, the header pages of each of
 the logical streams should be grouped together before any data pages
 occur.
%TBT: should or must?

After all the header pages have been placed,
 the data pages are multiplexed together.
They should be placed in the stream in increasing order by the playback
 time equivalents of their granule fields.
This facilitates seeking while limiting the buffering requirements of the
 playback demultiplexer.
%TODO: A lot of this language is encoder-oriented.
%TODO: We define a decoder-oriented specification.
%TODO: The language should be changed to match.

\section{Colophon}

%TODO: Logo

Ogg is a \href{http://www.xiph.org}{Xiph.org Foundation} effort to protect
 essential tenets of Internet multimedia from corporate hostage-taking; Open
 Source is the net's greatest tool to keep everyone honest.
See \href{http://www.xiph.org/about.html}{About the Xiph.org Foundation} for
 details.

Ogg Theora is the first Ogg video codec.
Anyone may freely use and distribute the Ogg and Theora specification, whether
 in private, public, or corporate capacity.
However, the Xiph.org Foundation and the Ogg project reserve the right to set
 the Ogg Theora specification and certify specification compliance.

Xiph.org's Theora software codec implementation is distributed under a BSD-like
 license.
This does not restrict third parties from distributing independent
 implementations of Theora software under other licenses.

Ogg, Theora, Vorbis, Xiph.org Foundation and their logos are trademarks (tm) of
 the \href{http://www.xiph.org}{Xiph.org Foundation}.
These pages are copyright \copyright{} 2004 Xiph.org Foundation.
All rights reserved.

This document is set in \LaTeX.

\bibliography{spec}

\end{document}
